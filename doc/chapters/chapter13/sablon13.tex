% Szglab4
% ===========================================================================
%
\chapter{Grafikus felület specifikációja}

\thispagestyle{fancy}

\section{Fordítási és futtatási útmutató}
%\comment{A feltöltött program fordításával és futtatásával kapcsolatos útmutatás. Ennek tartalmaznia kell leltárszerűen az egyes fájlok pontos nevét, méretét byte-ban, keletkezési idejét, valamint azt, hogy a fájlban mi került megvalósításra.}

\subsection{Fájllista}

\begin{fajllista}


\fajl
{icons/background.png}
{323033 byte}
{2014.05.03~19:15~}
{A játéktér háttérképe.}

\fajl
{icons/blue\_gem.png}
{2298 byte}
{2014.05.03~17:48~}
{A kék varázskövek ikonja.}

\fajl
{icons/dwarf.png}
{304 byte}
{2014.05.04~19:06~}
{A törpök ikonja.}

\fajl
{icons/elf.png}
{308 byte}
{2014.05.04~19:06~}
{A tündék ikonja.}

\fajl
{icons/green\_gem.png}
{2311 byte}
{2014.05.03~17:48~}
{A zöld varázskövek ikonja.}

\fajl
{icons/hobbit.png}
{319 byte}
{2014.05.04~19:06~}
{A hobbitok ikonja.}

\fajl
{icons/human.png}
{312 byte}
{2014.05.04~19:06~}
{Az emberek ikonja.}

\fajl
{icons/LZF.png}
{52032 byte}
{2014.05.05~16:29~}
{A játék megnyerésekor megjelenő kép.}

\fajl
{icons/obstacle.png}
{2430 byte}
{2014.05.03~17:48~}
{Az akadályok ikonja.}

\fajl
{icons/orange\_gem.png}
{2526 byte}
{2014.05.03~17:48~}
{A narancssárga varázskövek ikonja.}

\fajl
{icons/projectile.png}
{1167 byte}
{2014.05.04~19:22~}
{A lövedékek ikonja.}

\fajl
{icons/red\_gem.png}
{2302 byte}
{2014.05.03~17:48~}
{A piros varázskövek ikonja.}

\fajl
{icons/saurontower.png}
{43961 byte}
{2014.05.03~19:15~}
{A Végzet Hegyét és Szauron tornyát ábrázoló kép.}

\fajl
{icons/splitter\_projectile.png}
{905 byte}
{2014.05.04~19:22~}
{A kettévágó lövedékek ikonja.}

\fajl
{icons/tower.png}
{989 byte}
{2014.05.03~17:48~}
{A tornyok ikonja.}

\fajl
{icons/yellow\_gem.png}
{2443 byte}
{2014.05.03~17:48~}
{A sárga varázskövek ikonja.}


\fajl
{maps/demo.map}
{1228 byte}
{2014.05.04~20:42~}
{A felületi terven is látható pályafájl.}

\fajl
{maps/test.map}
{267 byte}
{2014.05.04~03:54~}
{Egy minimális pályafájl.}


\fajl
{missions/demo\_advanced.mission}
{5401 byte}
{2014.05.10~15:00~}
{Egy összetettebb missziófájl.}

\fajl
{missions/demo\_attack.mission}
{1251 byte}
{2014.05.04~20:42~}
{Egy egyszerű missziófájl.}

\fajl
{missions/demo\_delay.mission}
{1250 byte}
{2014.05.05~01:19~}
{Egy egyszerű missziófájl, hosszabb felkészülési idővel.}

\fajl
{missions/test\_attack.mission}
{139 byte}
{2014.05.04~03:54~}
{Egy minimális missziófájl.}



\fajl
{src/Controller.java}
{5456 byte}
{2014.05.03~15:20~}
{Az MVC modell controller részét valósítja meg.}

\fajl
{src/Drawable.java}
{2322 byte}
{2014.05.03~18:07~}
{A kirajzolható objektumok közös ősosztálya.}

\fajl
{src/Enemy.java}
{3362 byte}
{2014.03.20~20:33~}
{Az ellenségeket megvalósító osztály.}

\fajl
{src/EnemyType.java}
{998 byte}
{2014.03.21~12:57~}
{Az ellenségtípusokat leíró osztály.}

\fajl
{src/Fog.java}
{640 byte}
{2014.04.21~18:25~}
{A ködöt működtető osztály.}

\fajl
{src/Game.java}
{10064 byte}
{2014.03.20~09:55~}
{A játék logikáját összefogó osztály.}

\fajl
{src/Gem.java}
{409 byte}
{2014.03.21~12:57~}
{A varázskövek közös ősosztálya.}

\fajl
{src/GemButton.java}
{289 byte}
{2014.05.03~16:09~}
{A gombok, amelyekkel varázskövet lehet rakni az épületekre.}

\fajl
{src/GraphicEnemy.java}
{1205 byte}
{2014.05.03~18:07~}
{Az ellenségek kirajzolásáért felelős osztály.}

\fajl
{src/GraphicFog.java}
{432 byte}
{2014.05.03~21:32~}
{A köd kirajzolásáért felelős osztály.}

\fajl
{src/GraphicGem.java}
{1101 byte}
{2014.05.05~20:48~}
{A varázskövek kirajzolásáért felelős osztály.}

\fajl
{src/GraphicMap.java}
{2098 byte}
{2014.05.03~19:14~}
{A pálya kirajzolásáért felelős osztály.}

\fajl
{src/GraphicObstacle.java}
{2290 byte}
{2014.05.03~18:16~}
{Az akadályok kirajzolásáért felelős osztály.}

\fajl
{src/GraphicProjectile.java}
{1016 byte}
{2014.05.03~18:07~}
{A lövedékek kirajzolásáért felelős osztály.}

\fajl
{src/GraphicTower.java}
{2401 byte}
{2014.05.03~18:16~}
{A tornyok kirajzolásáért felelős osztály.}

\fajl
{src/Main.java}
{2645 byte}
{2014.05.04~19:32~}
{Az alkalmazást elindító osztály.}

\fajl
{src/Map.java}
{3801 byte}
{2014.03.20~20:33~}
{Egy pályát megvalósító osztály.}

\fajl
{src/Menu.java}
{6174 byte}
{2014.05.04~03:53~}
{A játék indító menüjét megvalósító osztály.}

\fajl
{src/Mission.java}
{2354 byte}
{2014.03.20~20:33~}
{Egy küldetést leíró osztály.}

\fajl
{src/Obstacle.java}
{2206 byte}
{2014.03.21~12:57~}
{Az akadályokat megvalósító osztály.}

\fajl
{src/ObstacleGem.java}
{805 byte}
{2014.03.20~20:33~}
{Az akadályokra tehető varázskövek.}

\fajl
{src/Pair.java}
{209 byte}
{2014.04.21~13:10~}
{Párokat tároló generikus osztály.}

\fajl
{src/Projectile.java}
{1219 byte}
{2014.03.20~20:33~}
{Egy lövedék viselkedését megvalósító osztály.}

\fajl
{src/Resources.java}
{1949 byte}
{2014.05.05~21:58~}
{Az összes játékban használt képet eltároló osztály.}

\fajl
{src/Spawn.java}
{302 byte}
{2014.04.22~00:59~}
{Egy konkrét ellenség típusát és játékba lépésének idejét tárolja.}

\fajl
{src/SplitterProjectile.java}
{729 byte}
{2014.04.21~17:43~}
{A kettévágó lövedékeket megvalósító osztály.}

\fajl
{src/Tower.java}
{3646 byte}
{2014.03.20~20:33~}
{Egy tornyot megvalósító osztály.}

\fajl
{src/TowerGem.java}
{902 byte}
{2014.03.20~20:33~}
{A tornyokra rakható varázskövek.}

\fajl
{src/Vector.java}
{1397 byte}
{2014.03.20~20:33~}
{Egy koordináta pontot megvalósító osztály.}

\fajl
{src/View.java}
{10510 byte}
{2014.05.03~14:31~}
{Az MVC modell view része.}

\fajl
{src/Waypoint.java}
{2208 byte}
{2014.03.21~12:57~}
{Az utakat kijelölő pontokat magvalósító osztály.}

\fajl
{src/Window.java}
{795 byte}
{2014.05.04~03:53~}
{A grafikus ablakot kezelő osztály.}

\end{fajllista}

\subsection{Fordítás}
%\comment{A fenti listában szereplő forrásfájlokból milyen műveletekkel lehet a bináris, futtatható kódot előállítani. Az előállításhoz csak a 2. Követelmények c. dokumentumban leírt környezetet szabad előírni.}

A fordításhoz lépjünk be egy terminálban a projekt főkönyvtárába, majd adjuk ki a következő parancsot:
\lstset{escapeinside=`', xleftmargin=10pt, frame=single, basicstyle=\ttfamily\footnotesize, language=sh}
\begin{lstlisting}
javac -d . -encoding utf8 src/szoftlab4/*.java
\end{lstlisting}

\subsection{Futtatás}
%\comment{A futtatható kód elindításával kapcsolatos teendők leírása. Az indításhoz csak a 2. Követelmények c. dokumentumban leírt környezetet szabad előírni.}
A futtatáshoz pedig a fordítást követően:
\lstset{escapeinside=`', xleftmargin=10pt, frame=single, basicstyle=\ttfamily\footnotesize, language=sh}
\begin{lstlisting}
java szoftlab4.Main
\end{lstlisting}

A fenti parancs után szóközzel elválasztva fűzhetünk még tetszés szerint kétféle parancssori argumentumot.
Egyikkel bekapcsolhatjuk az élsimítást (antialiasingot) a kirajzolásnál, a másikkal pedig a másodpercenként kirajzolt képkockák számát jeleníttethetjük meg a grafikus ablak jobb felső sarkában. Szokás szerint mindkét paraméternek van egy hosszú és egy rövid nevű változata.\\

A megszokott jelölést követve tehát a lehetséges opciók:

\begin{lstlisting}
java szoftlab4.Main [--antialiasing|-aa] [--showfps|-f]
\end{lstlisting}

\newpage

\section{Értékelés}
%\comment{A projekt kezdete óta az értékelésig eltelt időben tagokra bontva, százalékban.}

\begin{ertekeles}
\tag{Nusser} % Tag neve
{22}        % Munka szazalekban
\tag{Szabó}
{27}
\tag{Tallér}
{29}
\tag{Török}
{22}
\end{ertekeles}

