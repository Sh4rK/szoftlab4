% Szglab4
% ===========================================================================
%
\chapter{Szkeleton beadás}

\thispagestyle{fancy}

\section{Fordítási és futtatási útmutató}

\subsection{Fájllista}

\begin{fajllista}

\fajl
{Enemy.java} % Kezdet
{1 967 byte} % Idptartam
{2014.03.20~10:51~}
{Enemy osztály.} % Leírás

\fajl
{EnemyType.java} % Kezdet
{530 byte} % Idptartam
{2014.03.21~22:53~}
{EnemyType osztály.} % Leírás

\fajl
{Game.java} % Kezdet
{9 606 byte} % Idptartam
{2014.03.20~9:25~}
{Game osztály. Innen indul a program.} % Leírás

\fajl
{Gem.java} % Kezdet
{403 byte} % Idptartam
{2014.03.21~22:53~}
{Gem osztály.} % Leírás

\fajl
{Map.java} % Kezdet
{926 byte} % Idptartam
{2014.03.20~10:51~}
{Map osztály.} % Leírás

\fajl
{Mission.java} % Kezdet
{771 byte} % Idptartam
{2014.03.20~10:51~}
{Mission osztály.} % Leírás

\fajl
{Obstacle.java} % Kezdet
{1 693 byte} % Idptartam
{2014.03.21~22:53~}
{Obstacle osztály.} % Leírás

\fajl
{ObstacleGem.java} % Kezdet
{599 byte} % Idptartam
{2014.03.20~10:32~}
{ObstacleGem osztály.} % Leírás

\fajl
{Projectile.java} % Kezdet
{1 165 byte} % Idptartam
{2014.03.20~10:51~}
{Projectile osztály.} % Leírás

\fajl
{Tower.java} % Kezdet
{2 085 byte} % Idptartam
{2014.03.20~10:51~}
{Tower osztály.} % Leírás

\fajl
{TowerGem.java} % Kezdet
{578 byte} % Idptartam
{2014.03.20~10:32~}
{TowerGem osztály.} % Leírás

\fajl
{Vector.java} % Kezdet
{50 byte} % Idptartam
{2014.03.20~10:31~}
{Vector segédosztály.} % Leírás

\fajl
{Waypoint.java} % Kezdet
{853 byte} % Idptartam
{2014.03.21~22:53~}
{Waypoint osztály.} % Leírás

\end{fajllista}

\subsection{Fordítás}

\lstset{escapeinside=`', xleftmargin=10pt, frame=single, basicstyle=\ttfamily\footnotesize, language=sh}
\begin{lstlisting}
javac -d bin src/szoftlab4/*.java
\end{lstlisting}

\subsection{Futtatás}

\lstset{escapeinside=`', xleftmargin=10pt, frame=single, basicstyle=\ttfamily\footnotesize, language=sh}
\begin{lstlisting}
cd bin
java szoftlab4.Game
\end{lstlisting}

\section{Értékelés}

\begin{ertekeles}
\tag{Nusser} % Tag neve
{25}        % Munka szazalekban
\tag{Szabó}
{25}
\tag{Tallér}
{25}
\tag{Török}
{25}
\end{ertekeles}

\section{Módosítások}
\subsection{Menü}
A 7. Varázskő felrakása menüpontba raktunk egy *7.2 Érvényes helyet adtunk meg? kérdést.
Egy új menüpontot raktunk a programba \\
8. Ellenségek lassítása \\
*8.1 Van varázskő az akadályon (I/N)? \\
*8.2 Van ellenség az akadály hatókörében (I/N)? \\
Ezzel együtt a 8. Kilépés menüpont, 9. Kilépésre változott. \\

\subsection{Game osztály}
A Game.run metódus visszatérési értékét boolean-ra változtattuk. Igazzal tér vissza ha a játékos nyert/vesztett/feladata, hamissal tér vissza ha kilép a játékból.

\subsection{Obstacle}
Az Obstacle-ből kimaradt a getRange() metódus, ami az osztálydiagramban és az osztály leírásánál nem szerepel, de a szekvencia diagramokban igen. A metódus visszaadja az akadály hatótávolságát.
