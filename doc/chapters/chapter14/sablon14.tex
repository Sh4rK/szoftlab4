% Szglab4
% ===========================================================================
%
\chapter{Összefoglalás}

\thispagestyle{fancy}

\section{Projekt összegzés}
%\comment{A projekt tapasztalatait összegző részben a csapatoknak a projektről kialakult véleményét várjuk. A megválaszolandók köre az alábbi. }

\begin{munka}
\munkaido{Nusser}{46}
\munkaido{Szabó}{54.5}
\munkaido{Tallér}{58.5}
\munkaido{Török}{46.5}
\osszesmunkaido{205.5}
\end{munka}

\begin{forrassor}
\munkaido{Szkeleton}{962}
\munkaido{Protó}{1686}
\munkaido{Grafikus}{2993}
\end{forrassor}

\begin{itemize}

\item Mit tanultak a projektből konkrétan és általában? \newline
Konkrétan a \LaTeX{} nyelv alapjait, valamint git használatát, és annak segítségével munka megosztását DVCS-en keresztül, ugyanis eddig is használtam már verziókezelő rendszert, de nem gitet, és csak egyedül. Általában pedig azt, hogy egy nagyobb szoftver megtervezése és lefejlesztése, dokumentálása vajon hogyan is történhet(ett), még ha ez a projekt talán - László Zoltán tanár úr szavaival élve - "kutyaólépítés toronydaruval" volt is.

\item Mi volt a legnehezebb és a legkönnyebb? \newline
A legnehezebbnek a temérdek dokumentáció és számtalan diagram legyártását tartottam egy még nem létező dologról. Valahogy jobban tetszenek az iteratív fejlesztési módszerek, amik használatakor legalább a dokumentációval együtt fejlődik a program, még ha nem is a végleges verziója, de egy működő, futó prototípusa annak, és nem ilyen lineáris a dokumentációtól a kód felé haladás. Főleg a korai szakaszban való túlzottan részletekbe bocsátkozás kényszerítése idegenkedett tőlem, ugyanis úgy vélem, hogy az utolsó függvényparaméterig lehetetlen úgy megtervezni egy szoftvert, még egy ilyen kicsit is, hogy az végül illeszkedjen az eleinte elképzeltekre. Ha pedig a realizálódás során úgyis változni fognak a részletek, akkor egyáltalán miért írnánk le róluk egy légből kapott elképzelést? Persze a gondos, de viszonylag nagy vonalakban dolgozó architektúrális tervezést én is elengedhetetlennek tartom már a kezdetektől fogva.
Legkönnyebb nekem maga a lényegi kódolás volt, ugyanis abban már van gyakorlatom, és szívesen is csinálom, mert akkor érzem csak úgy, mintha valami működőt teremtenék.

\item Összhangban állt-e az idő és a pontszám az elvégzendő feladatokkal? \newline
A tárgyon belül kiosztott pontokat nem követtem minden feladatrésznél egészen pontosan nyomon, ezért erre nem tudok megfontolt választ adni, de az egyes csapattagok részvételi aránya és lelkesedése hétről hétre erősen ingadozhat, ezért a teher tetszés szerint megosztható.

\item Milyen változtatási javaslatuk van? \newline
Irreális követelménynek tartom a 6-os verziójú Java használatát, ugyanis ez már nem letölthető a gyártó hivatalos honlapjáról, ilyen régi JDK-t pedig szinte lehetetlen bárhol máshol találni, és JRE-t is csak kétséges forrásokból sikerült. Arra való tekintettel pedig, hogy nemrégiben megjelent a 8-as verzió, nem tartanám kizártnak, hogy egy év múlva már a 7-es sem lesz egykönnyen beszerezhető a HSZK pincéjében féltve őrzött lyukkártyáktól különböző médiumról.
Apróbb kellemetlenség volt még számomra a rendszeres hétfő reggeli beadási határidő.

\item Milyen feladatot ajánlanának a projektre? \newline
Nagyszerű ötletnek tartanék egy síkbeli Minecraft-szerű, "ásós-szerkesztős-építős" játékot.

\item Egyéb észrevételek, kritikák

Egy kicsit neheztelem azt is (igaz, legjobb tudomásom szerint ez már nem az Önök hatásköre, ezért csak mint véleményt, és nem mint kritikát írom), hogy ezért az 50-60 óra tényleges munkáért mindössze két kredit jár, ami - a legkisebb túlzás nélkül - mindenféle mókás kötelezően választható tárgyakból egyetlen délután alatt megszerezhető, némi szerencsével akár 4-es, vagy jobb érdemjeggyel. Nem magát az abszolút mértéket tartom inkorrektnek, tehát nem az a kérdés, hogy 1 kredit épp 5 vagy 150 órányi munkát jelképez elméletben, hanem abban látom a problémát, hogy a tárgyak között hatalmas szórás van az egy kreditért elvégzendő munkában, holott a későbbiekben már teljesen egyformán van kezelve, attól függetlenül, hogy milyen tantárgyból származik. Pedig szerintem az vitathatatlan, hogy a két órányi előadásra beüléshez (esetleg az arról való ellógáshoz) szükséges erőfeszítés meg sem közelíti a két órányi szekvenciadiagram-rajzoláshoz kellőt.

\end{itemize}

%\comment{Szívesen fogadunk minden egyéb kritikát és javaslatot.}

