% Szglab4
% ===========================================================================
%
\chapter{Összefoglalás}

\thispagestyle{fancy}

\section{Projekt összegzés}
%\comment{A projekt tapasztalatait összegző részben a csapatoknak a projektről kialakult véleményét várjuk. A megválaszolandók köre az alábbi. }

\begin{munka}
\munkaido{Nusser}{46}
\munkaido{Szabó}{54.5}
\munkaido{Tallér}{58.5}
\munkaido{Török}{46.5}
\osszesmunkaido{205.5}
\end{munka}

\begin{forrassor}
\munkaido{Szkeleton}{962}
\munkaido{Protó}{1686}
\munkaido{Grafikus}{2993}
\end{forrassor}


\subsection{Nusser Ádám véleménye}
\begin{itemize}
\item Mit tanultak a projektből konkrétan és általában?\newline
	Ahogy hallottam, az iparban nincs mindig idő (soha), hogy ilyen részletesen megtervezzünk és dokumentáljunk egy programot, így jó volt legalább egyszer végigmenni a lépésin. Ezzel kicsit betekintést nyertünk szerintem a szoftveresek világába amiből sokat tanultunk. A módszert ismertük, de így, hogy a gyakorlatban is láttuk sokat segített abban, hogy valóban értelmét lássuk az egésznek. Hasznos volt még megismerni külömböző programokat és nyelveket (Github, LateX,...) melyek a jövőben is hasznosak lehetnek.
\newline Talán csapatmunkával kapcsolatban tanultuk a legkevesebbet, mert jól programozik és szorgalmas a csapat, így az egyéni munka sokkal több volt mint a csapatmunka, és míg ez nálunk működött, nagyon rossz vége is lehetett volna.
\item Mi volt a legnehezebb és a legkönnyebb? \newline
	A programváz tervezése volt talán a legnehezebb, hiszen nagyon nehéz olyan architektúrát tervezni, mely majd minden változtatás, és minden hozzáadott dolgot túléljen, és működjön nagyobb változtatások nélkül. A csapatban való dolgozás sem lett volna egyszerű, ha a csapattársaim nem lennének ilyen kiváló szakemberek, mert sajnos sokszor egyéni munkával oldottunk meg olyan feladatokat, melyeket sokkal egyszerűbben meg tudtunk volna oldani, ha leülünk és részletesebben átgondoljuk megbeszéljük (ez főleg a vége felé igaz). 
\newline A legegyszerűbb rész az implementáció volt, hiszen hetek óta azt terveztük és mire oda értünk hogy kódolni kéne, már kentük-vágtuk, hogy mit fogunk nagyjából írni. 
\item Összhangban állt-e az idő és a pontszám az elvégzendő feladatokkal? \newline
	Nagyrészt igen. Volt egy-két rész, amit ha az ember előtte jól megírt akkor később alig volt vele munka és mégis sok pontot ért, de ez így jó.
\item Ha nem, akkor hol okozott ez nehézséget? \newline
	Nem okozott.
\item Milyen változtatási javaslatuk van? \newline
	Szerintem a tervezéskor van egy-két olyan feladat, mely nagyon kicsi módosításokkal már szerepelt korábban, és így érdemi munkát nem ad, csak megy vele az idő (Tudom, hogy a RUP része, de akkor is kihagyható lenne). Másrészt viszont a tesztelésre jobban kitérhetnénk, és kicsit jobban specifikált teszt feladatot kaphatnánk, mert az nem kap véleményem szerint annyi odafigyelést mint amennyit érdemelne. 
\item Milyen feladatot ajánlanának a projektre? \newline
	Esetleg felül-nézetes, Rouge-like RPG-t lehetne feladni még a jövőben, mert nem túl nehéz megírni egy olyat, de jól meg kell tervezni, hogy könnyen bővíthető legyen.
\end{itemize}

\subsection{Tallér Bátor véleménye}
\begin{itemize}
\item Mit tanultak a projektből konkrétan és általában? \newline
Csapatban dolgozni nehéz, de egyben rengeteg dolgot megkönnyít. Ha valakire nincs kiadva feladat akkor az a feladat nem lesz megcsinálva. 
A sok tervezés és dokumentáció után már könnyű azt leprogramozni.
\item Mi volt a legnehezebb és a legkönnyebb? \newline
A legnehezebb része a tervezés volt, konkrétan azon belül az, amikor kitaláltunk valamit majd később rájöttünk, hogy ez valamilyen ok miatt mégse megfelelő, és ezért újra kellett tervezni egyes részeket.\\
Továbbá nehézséget okozott összehangolni a projektet és az időnket.
\item Összhangban állt-e az idő és a pontszám az elvégzendő feladatokkal? \newline
Nagyjából összhangban állt. A súlyozással viszont nem értek egyet, mert pont a prototípussal dolgoztunk a legkevesebbet, míg a szkeletonnál sokat terveztünk és a grafikusnál meg sokat kódoltunk.\\
\item Ha nem, akkor hol okozott ez nehézséget? \newline
Nem okozott nehézségeket.
\item Milyen változtatási javaslatuk van? \newline
A dokumentáció sablonokat, illetve a tesztelési dokumentációkat aktualizálni kéne. A tesztelősben konkrétan ott áll az xls tetején, hogy 2009/2010, továbbá olyan kérdések voltak benne, amik nem tartoztak az aktuális feladathoz.\\\\
Véleményem szerint egy agilis módszertan jobb lenne a RUP helyett. Néhányszor azt éreztem, hogy a dokumentáció egyes részei feleslegesek (pl. állapotdiagram).\\\\
Hercules fejlesztése. Az az oldal azon kívül, hogy feltöltsük a beadandókat semmire se jó. Legalább az elért pontszámainkat feltüntethetné.
\item Milyen feladatot ajánlanának a projektre? \newline
Konkrét feladat nem jut az eszembe, de az ideihez hasonló teljesen megfelel a célra, mert nem csak egy unalmas program készül a végén, hanem egy működő játék.

\end{itemize}


\subsection{Török Attila véleménye}
\begin{itemize}

\item Mit tanultak a projektből konkrétan és általában? \newline
Konkrétan a \LaTeX{} nyelv alapjait, valamint git használatát, és annak segítségével munka megosztását DVCS-en keresztül, ugyanis eddig is használtam már verziókezelő rendszert, de nem gitet, és csak egyedül. Általában pedig azt, hogy egy nagyobb szoftver megtervezése és lefejlesztése, dokumentálása vajon hogyan is történhet(ett), még ha ez a projekt talán - László Zoltán tanár úr szavaival élve - "kutyaólépítés toronydaruval" volt is.

\item Mi volt a legnehezebb és a legkönnyebb? \newline
A legnehezebbnek a temérdek dokumentáció és számtalan diagram legyártását tartottam egy még nem létező dologról. Valahogy jobban tetszenek az iteratív fejlesztési módszerek, amik használatakor legalább a dokumentációval együtt fejlődik a program, még ha nem is a végleges verziója, de egy működő, futó prototípusa annak, és nem ilyen lineáris a dokumentációtól a kód felé haladás. Főleg a korai szakaszban való túlzottan részletekbe bocsátkozás kényszerítése idegenkedett tőlem, ugyanis úgy vélem, hogy az utolsó függvényparaméterig lehetetlen úgy megtervezni egy szoftvert, még egy ilyen kicsit is, hogy az végül illeszkedjen az eleinte elképzeltekre. Ha pedig a realizálódás során úgyis változni fognak a részletek, akkor egyáltalán miért írnánk le róluk egy légből kapott elképzelést? Persze a gondos, de viszonylag nagy vonalakban dolgozó architektúrális tervezést én is elengedhetetlennek tartom már a kezdetektől fogva.
Legkönnyebb nekem maga a lényegi kódolás volt, ugyanis abban már van gyakorlatom, és szívesen is csinálom, mert akkor érzem csak úgy, mintha valami működőt teremtenék.

\item Összhangban állt-e az idő és a pontszám az elvégzendő feladatokkal? \newline
A tárgyon belül kiosztott pontokat nem követtem minden feladatrésznél egészen pontosan nyomon, ezért erre nem tudok megfontolt választ adni, de az egyes csapattagok részvételi aránya és lelkesedése hétről hétre erősen ingadozhat, ezért a teher tetszés szerint megosztható.

\item Milyen változtatási javaslatuk van? \newline
Irreális követelménynek tartom a 6-os verziójú Java használatát, ugyanis ez már nem letölthető a gyártó hivatalos honlapjáról, ilyen régi JDK-t pedig szinte lehetetlen bárhol máshol találni, és JRE-t is csak kétséges forrásokból sikerült. Arra való tekintettel pedig, hogy nemrégiben megjelent a 8-as verzió, nem tartanám kizártnak, hogy egy év múlva már a 7-es sem lesz egykönnyen beszerezhető a HSZK pincéjében féltve őrzött lyukkártyáktól különböző médiumról.
Apróbb kellemetlenség volt még számomra a rendszeres hétfő reggeli beadási határidő.

\item Milyen feladatot ajánlanának a projektre? \newline
Nagyszerű ötletnek tartanék egy síkbeli Minecraft-szerű, "ásós-szerkesztős-építős" játékot.

\item Egyéb észrevételek, kritikák

Egy kicsit neheztelem azt is (igaz, legjobb tudomásom szerint ez már nem az Önök hatásköre, ezért csak mint véleményt, és nem mint kritikát írom), hogy ezért az 50-60 óra tényleges munkáért mindössze két kredit jár, ami - a legkisebb túlzás nélkül - mindenféle mókás kötelezően választható tárgyakból egyetlen délután alatt megszerezhető, némi szerencsével akár 4-es, vagy jobb érdemjeggyel. Nem magát az abszolút mértéket tartom inkorrektnek, tehát nem az a kérdés, hogy 1 kredit épp 5 vagy 150 órányi munkát jelképez elméletben, hanem abban látom a problémát, hogy a tárgyak között hatalmas szórás van az egy kreditért elvégzendő munkában, holott a későbbiekben már teljesen egyformán van kezelve, attól függetlenül, hogy milyen tantárgyból származik. Pedig szerintem az vitathatatlan, hogy a két órányi előadásra beüléshez (esetleg az arról való ellógáshoz) szükséges erőfeszítés meg sem közelíti a két órányi szekvenciadiagram-rajzoláshoz kellőt.

\end{itemize}


