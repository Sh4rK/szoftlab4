% Szglab4
% ===========================================================================
%
\chapter{Összefoglalás}

\thispagestyle{fancy}

\section{Projekt összegzés}
\comment{A projekt tapasztalatait összegző részben a csapatoknak a projektről kialakult véleményét várjuk. A megválaszolandók köre az alábbi. }

\begin{munka}
\munkaido{Horváth}{98}
\munkaido{Németh}{95}
\munkaido{Tóth}{102}
\munkaido{Oláh}{87}
\osszesmunkaido{382}
\end{munka}

\begin{forrassor}
\munkaido{Szkeleton}{500}
\munkaido{Protó}{600}
\munkaido{Grafikus}{700}
\end{forrassor}

\begin{itemize}
\item Mit tanultak a projektből konkrétan és általában?\newline
	
\item Mi volt a legnehezebb és a legkönnyebb? \newline
	\opinion{\adam}{A programváz tervezése volt talán a legnehezebb, hiszen nagyon nehéz olyan architektúrát tervezni, mely majd minden változtatás, és minden hozzáadott dolgot túléljen, és működjön nagyobb változtatások nélkül. A csapatban való dolgozás sem lett volna egyszerű, ha a csapattársaim nem lennének ilyen kiváló szakemberek, mert sajnos sokszor egyéni munkával oldottunk meg olyan feladatokat, melyeket sokkal egyszerűbben meg tudtunk volna oldani, ha leülünk és részletesebben átgondoljuk megbeszéljük (ez főleg a vége felé igaz). 
\newline A legegyszerűbb rész az implementáció volt. }
\item Összhangban állt-e az idő és a pontszám az elvégzendő feladatokkal? \newline
	\opinion{\adam}{Nagyrészt igen. Volt egy-két rész, amit ha az ember előtte jól megírt akkor később alig volt vele munka és mégis sok pontot ért, de ez így jó.}
\item Ha nem, akkor hol okozott ez nehézséget? \newline
\item Milyen változtatási javaslatuk van? \newline
\item Milyen feladatot ajánlanának a projektre? \newline
\end{itemize}

\comment{Szívesen fogadunk minden egyéb kritikát és javaslatot.}

