% Szglab4
% ===========================================================================
%
\chapter{Összefoglalás}

\thispagestyle{fancy}

\section{Projekt összegzés}
\comment{A projekt tapasztalatait összegző részben a csapatoknak a projektről kialakult véleményét várjuk. A megválaszolandók köre az alábbi. }

\begin{munka}
\munkaido{Horváth}{98}
\munkaido{Németh}{95}
\munkaido{Tóth}{102}
\munkaido{Oláh}{87}
\osszesmunkaido{382}
\end{munka}

\begin{forrassor}
\munkaido{Szkeleton}{500}
\munkaido{Protó}{600}
\munkaido{Grafikus}{700}
\end{forrassor}

\begin{itemize}
\item Mit tanultak a projektből konkrétan és általában?\newline
	\opinion{\adam}{Fasza.}
\item Mi volt a legnehezebb és a legkönnyebb? \newline
\item Összhangban állt-e az idő és a pontszám az elvégzendő feladatokkal? \newline
\item Ha nem, akkor hol okozott ez nehézséget? \newline
\item Milyen változtatási javaslatuk van? \newline
\item Milyen feladatot ajánlanának a projektre? \newline

\subsection{Tallér Bátor véleménye}
\begin{itemize}
\item Mit tanultak a projektből konkrétan és általában? \newline
Csapatban dolgozni nehéz, de egyben rengeteg dolgot megkönnyít. Ha valakire nincs kiadva feladat akkor az a feladat nem lesz megcsinálva. A sok tervezés és dokumentáció után már könnyű azt leprogramozni.
\item Mi volt a legnehezebb és a legkönnyebb? \newline
A legnehezebb része a tervezés volt, konkrétan azon belül az, amikor kitaláltunk valamit majd később rájöttünk, hogy ez valamilyen ok miatt mégse megfelelő, és ezért újra kellett tervezni egyes részeket.\\
Továbbá nehézséget okozott összehangolni a projektet és az időnket.
\item Összhangban állt-e az idő és a pontszám az elvégzendő feladatokkal? \newline
Nagyjából összhangban állt. A súlyozással viszont nem értek egyet, mert pont a prototípussal dolgoztunk a legkevesebbet, míg a szkeletonnál sokat terveztünk és a grafikusnál meg sokat kódoltunk.\\
\item Ha nem, akkor hol okozott ez nehézséget? \newline
Nem okozott nehézségeket.
\item Milyen változtatási javaslatuk van? \newline
A dokumentáció sablonokat, illetve a tesztelési dokumentációkat aktualizálni kéne. A tesztelősben konkrétan ott áll az xls tetején, hogy 2009/2010, továbbá olyan kérdések voltak benne, amik nem tartoztak az aktuális feladathoz.\\\\
Véleményem szerint egy agilis módszertan jobb lenne a RUP helyett. Néhányszor azt éreztem, hogy a dokumentáció egyes részei feleslegesek (pl. állapotdiagram).\\\\
Hercules fejlesztése. Az az oldal azon kívül, hogy feltöltsük a beadandókat semmire se jó. Legalább a elért pontszámainkat feltüntethetné.
\item Milyen feladatot ajánlanának a projektre? \newline
Konkrét feladat nem jut az eszembe, de az ideihez hasonló teljesen megfelel a célra, mert nem csak egy unalmas program készül a végén, hanem egy működő játék.
\end{itemize}

\end{itemize}
\comment{Szívesen fogadunk minden egyéb kritikát és javaslatot.}

