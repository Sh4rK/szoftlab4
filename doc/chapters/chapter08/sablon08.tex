% Szglab4
% ===========================================================================
%
\chapter{Részletes tervek}

\thispagestyle{fancy}

\section{Osztályok és metódusok tervei}

\subsection{Enemy}
\begin{itemize}
\item Felelősség\\
Az ellenségek pozíciójának, sebességének és életerejének nyilvántartása.
\item Attribútumok
	\begin{itemize}
		\item - \underline{num: int}: Az osszes ellenség számát tartja nyilván.
		\item - type: \textbf{EnemyType}: Az egység típusa.
		\item - health: Double: Az ellenség fennmaradó életereje.
		\item - position: Vector: Pillanatnyi pozíció a pályán.
		\item - targetWaypoint: \textbf{Waypoint}: Az a \textbf{Waypoint}, ami felé az ellenség jelenleg tart.
		\item - nextWaypoint: \textbf{Waypoint}: A következő \textbf{Waypoint} ami felé tartani fog.
		\item - slowingFactor: Double: Ha az ellenség belelép egy akadályba, akkor beállítja ennek az értékét. Az \textbf{EnemyType} sebességét ezzel kell beszorozni, hogy megkapjuk az ellenség tényleges jelenlegi sebességét.
		\item ID: int: Identifikálja az egyes ellenségeket.
	\end{itemize}
\item Metódusok
	\begin{itemize}
		\item + Enemy(et: EnemyType, start: Waypoint): Létrehoz egy \textbf{EnemyType} típusú \textbf{Enemy}-t a start \textbf{Waypoint}-helyén.
		\item + Enemy(type: EnemyType, start: Waypoint, ID: int): Megadott ID-val hoz létre ellenséget.
		\item + Enemy(en: Enemy): Copy konstruktor.
		\item + move(): boolean: Az ellenséget a sebességének megfelelő mértékben mozgatja a célja irányába. Ha az ellenség életereje 0 vagy kisebb, akkor igazzal tér visza, egyébként hamissal.
		\item + damage(amount: Double): boolean: Csökkenti az ellenség életerejét amount-al, igazzal tér vissza ha az ellenség meghalt.
		\item + getPosition(): Vector: A position attribútum értékével tér vissza.
		\item + getEnemyType(): \textbf{EnemyType}: Visszaadja az ellenség típusát.
		\item + getID(): int: Visszaadja az ellenség ID-ját.
		\item + getDistance(): Double: Visszaadja az ellenség céltól való távolságát a legrövidebb úton haladva.
		\item + setSlowingFactor(sf: Double): void: Beállítja a slowingFactor-t sf-re.
		\item + setNextWaypoint(w: \textbf{Waypoint}): void: Beállítja a következő \textbf{Waypoint}-ot.
		\item + split(d: Double): Enemy: Megsebzi az ellenséget d-vel, majd klónozza az objektumot és ezzel tér vissza.
	\end{itemize}
\end{itemize}


\subsection{EnemyType}
\begin{itemize}
\item Felelősség\\
Leírja egy bizonyos típusú (fajú) ellenség alapvető tulajdonságait. Egy-egy példányára hivatkozik az összes ellenség, amelyeknek ezáltal meghatározza a viselkedését. Az osztályból nem lehet példányosítani, csakis a statikus tagként szereplő objektumokat lehet felhasználni.
\item Attribútumok
	\begin{itemize}
		\item - magic: int: Az ilyen fajtájú ellenségek ára varázserőben.
		\item - initialHealth: Double: Az ilyen fajtájú ellenségek kezdeti életereje.
		\item - normalSpeed: Double: Az ilyen fajtájú ellenségek akadályoztatás nélküli haladási sebessége.
		\item + \underline{EnemyType dwarf}: Csak olvasható törp típus.
		\item + \underline{EnemyType elf}: Csak olvasható tünde típus.
		\item + \underline{EnemyType hobbit}: Csak olvasható hobbit típus.
		\item + \underline{EnemyType human}: Csak olvasható ember típus.
	\end{itemize}
\item Metódusok
	\begin{itemize}
		\item + getHealth(): Double: Visszaadja az initialHealth attribútum értékét.
		\item + getSpeed(): Double: Visszaadja a normalSpeed attribútum értékét.
	\end{itemize}
\end{itemize}

\subsection{Fog}
\begin{itemize}
\item Felelősség\newline
Ez az osztály felelős a ködért, ami a 7.0 fejezetben lett specifikálva. Az osztály statikus metódusokkal biztosítja a köd működését és ki- és bekapcsolását.
\item Attribútumok
	\begin{itemize}
		\item - \underline{isSet: boolean}: Ez tárolja, hogy be van-e kapcsolva a játékban a köd.
	\end{itemize}
\item Metódusok
	\begin{itemize}
		\item + \underline{getRangeMultiplier(): Double}: Ha be van kapcsolva a köd akkor egy <1 számmal tér vissza, amivel csökkenti a tornyok látótávát, ha nincs bekapcsolva akkor 1-el tér vissza.
		\item + \underline{setFog(b: boolean): void}: b paraméter értékére állítja az isSet attribútumot.
	\end{itemize}
\end{itemize}

\subsection{Game}
\begin{itemize}
\item Felelősség\\
A többi osztály nyilvántartása és összekötése, a játékbeli események vezérlése. A felhasználói felülettől érkező parancsok végrehajtása, és a játék állapotának rendelkezésre bocsájtása a kijelzéshez.
\item Attribútumok
	\begin{itemize}
		\item \underline{FPS: int}: Másodpercenként hányszor fut le a játék főciklusa.
		\item - map: \textbf{Map}: Referencia a kiválasztott pályára, amin a játék folyik.
		\item - mission: \textbf{Mission}: Referencia a kiválasztott misszióra, amely alapján zajlik a játék.
		\item - enemies: List<\textbf{Enemy}>: Az összes jelenleg élő ellenség található meg benne.
		\item - projectiles: List<\textbf{Projectile}>: Eltárolja a jelenleg játékban lévő lövedékeket.
		\item - towers: List<\textbf{Tower}>: Eltárolja a játékos megépített tornyait.
		\item - obstacles: List<\textbf{Obstacle}>: Eltárolja a játékos megépített akadályait.
		\item - magic: int: A játékos jelenlegi varázsereje.
	\end{itemize}
\item Metódusok
	\begin{itemize}
		\item + run(): boolean: Ez a metódus futtatja a főciklust, amelyben maga a játék működik. Ez a metódus hívja meg az ellenségek, lövedékek léptető metódusait. Meghívja a tornyok attack metódusát az ellenségek listájával. Az ellenségeknek beállítja a lassítást, ha akadályba léptek.
		\item + buildTower(position: Vector): void: Épít egy tornyot a paraméterül kapott helyen lévő mezőre, ha a pozíció a pályán belül, nem úton van nem ütközik másik toronnyal.
		\item + buildObstacle(position: Vector): void: Épít egy akadályt a paraméterül kapott helyen lévő mezőre, ha a megadott pozíció úton, pályán belül van és nem ütközik másik akadállyal..
		\item + addGem(position: Vector, gem: \textbf{TowerGem}): void: A paraméterként kapott helyen lévő toronyra rárakja a paraméterként kapott varázskövet.
		\item + addGem(position: Vector, gem: \textbf{ObstacleGem}): void: A paraméterként kapott helyen lévő akadályra rárakja a paraméterként kapott varázskövet.
		\item + addEnemy(en: Enemy): void: Hozzáadja a paraméterként kapott ellenséget az enemies listába.
		\item + collidesWithTower(p: Vector): boolean: megadja, hogy ha p helyre építenénk tornyot, az belelógna-e egy már megépített toronyba.
		\item +getCollidingTower(pos: Vector): Tower: Visszatér a pos helyén levő toronnyal.
		\item + collidesWithObstacle(p: Vector): boolean: megadja, hogy ha p helyre építenénk akadályt, az belelógna-e egy már megépített akadályba.
		\item + getCollidingObsacle(pos: Vector): Obstacle: Visszatér a pos helyén levő akadállyal.
		\item + getMagic(): int: Visszatér a magic-el.
		\item - getEnemyById(enemyID: int): Enemy: ID alapján visszatér az ellenséggel.
		\item + giveup(): void: A játék feladása.
		\item + slowEnemies(): void: Lelassítja az ellenségeket.
		\item - step(): boolean: A játék logikáját egy lépéssel előrébb viszi. Igazzal tér vissza ha sikeres volt.
	\end{itemize}
\end{itemize}

\subsection{Gem}
\begin{itemize}
\item Felelősség\\
Egy általános varázskő tulajdonságainak tárolása. Absztrakt osztály.
\item Attribútumok
	\begin{itemize}
		\item - cost: int: A varázskő ára varázserőben.
		\item - rangeMultiplier: Double: Megadja, hogy a varázskővel ellátott toronynak hányszorosára nő a hatótávolsága.
	\end{itemize}
\item Metódusok
	\begin{itemize}
		\item + getRangeMultiplier(): Double: Visszaadja a varázskő hatótávolság szorzóját.
	\end{itemize}
\end{itemize}

\subsection{Projectile}
\begin{itemize}
\item Felelősség\\
Követni a cél ellenséget, majd sebezni ha eléri.
\item Attribútumok
	\begin{itemize}
		\item - damage: Double: A lövedék sebzése, ennyivel csökkenti a cél ellenség életerejét amikor eltalálja.
		\item - position: Vector: A lövedék pozíciója.
		\item - speed: Double: A lövedék sebessége.
		\item - target: \textbf{Enemy}: A lövedék cél \textbf{Enemy}-je.
	\end{itemize}
\item Metódusok
	\begin{itemize}
		\item + Projectile(\textbf{Enemy} enemy, Vector position, double speed): Konstruktor, átveszi a cél \textbf{Enemy}-t, a kezdő pozíciót és sebességet.
		\item + step(): boolean: speed-el mozgatja a lövedéket az ellenség irányába. Ha eltalálta az ellenséget vagy az ellenség már meghalt, akkor true-t ad vissza, különben false-t.
		\item + getPosition(): Vector: Visszaadja a lövedék pozícióját.
	\end{itemize}
\end{itemize}

\subsection{SplitterProjectile}
\begin{itemize}
\item Felelősség\\
Követni a cél ellenséget, majd kettévágni ha eléri.
\item Ősosztályok\newline
Projectile
\item Attribútumok
	\begin{itemize}
		\item - game: Game: Egy referencia a játék objektumra. Erre a Game.addEnemy callback miatt van szükség.
	\end{itemize}
\item Metódusok
	\begin{itemize}
		\item + Projectile(\textbf{Enemy} enemy, Vector position, double speed, Game game): Konstruktor, átveszi a cél \textbf{Enemy}-t, a kezdő pozíciót, sebességet és egy referenciát a játékra.
	\end{itemize}
\end{itemize}

\subsection{Waypoint}
\begin{itemize}
\item Felelősség\\
Útvonalat kijelölni az ellenségeknek, úgy, hogy megadja a pozícióját, amely felé az ellenségek mehetnek, valamint a következő  \textbf{Waypoint}-ot ami felé menniük kell, ha egyszer elérték ezt a  \textbf{Waypoint}-ot.
\item Attribútumok
	\begin{itemize}
		\item - position: Vector: A \textbf{Waypoint} pozíciója a pályán.
		\item - distance: double: A  \textbf{Waypoint} távolságát a céltól tárolja.
		\item ID: int: Identifikálja az egyes \textbf{Waypoint}-okat.
		\item - nextWaypoints: List <Waypoint, double>: A következő  \textbf{Waypoint}-okat és a hozzájuk tartozó valószínűségeket
	\end{itemize}
\item Metódusok
	\begin{itemize}
		\item + Waypoint(pos: Vector, ID: int): Konstruktor, mely adott helyen adott ID-val létrehoz egy \textbf{Waypoint}-ot.
		\item + getDistance():double: Visszaadja a distance attribútumot.
		\item + getNextWaypoint(): \textbf{Waypoint}: Visszatér a nextWaypoints listából véletlenszerűen kiválasztott \textbf{Waypoint}-al
		\item + getPosition(): Vector: visszatér a position attribútummal.
		\item + getID(): int: Visszatér az ID-val.
		\item + listNextWaypoints(): List<\textbf{Waypoint}>: A nextWaypoints-ot listaként adja vissza.
		\item + setDistance(): double: Rekurzívan bejárja a pályát, és beállítja a céltól való távolságukat.
		\item + setID(ID: int): void: Beállítja az ID-t a kapott értékre.
		\item + setNextWaypoint(wp: \textbf{Waypoint}, d: double): void: Hozzáadja a nextWaypointokhoz a wp-t d valószinűséggel.
	\end{itemize}
\end{itemize}

\subsection{Map}
\begin{itemize}
\item Felelősség\newline
%\comment{Mi az osztály felelőssége. Kb 1 bekezdés. Ha szükséges, akkor state-chart is.}
Betölt egy XML pályaleíró fájlt, és ez alapján felépíti a pályát Waypointok-ból; elérhetővé teszi a Waypoint-okat id alapján; ellenőrizni tudja, hogy egy adott helyre építhető-e torony vagy akadály.
\item Attribútumok\newline
%\comment{Milyen attribútumai vannak}
	\begin{itemize}
		\item - \underline{roadRadius: double}: Az utak átmérője.
		\item - waypoints: HashMap<Integer, Waypoint>: A Waypoint-okat tárolja id alapján.
	\end{itemize}
\item Metódusok\newline
%\comment{Milyen publikus, protected és privát  metódusokkal rendelkezik. Metódusonként precíz leírás, ha szükséges, activity diagram is  a metódusban megvalósítandó algoritmusról.}
	\begin{itemize}
		\item Map(file: String): Létrehoz egy Map osztályt a paraméterként megadott pályaleíró fájlból.
		\item IsInRoadRange(pos: Vector, range: double): boolean: Visszatér azzal, hogy a pos úton van-e.
		\item getWaypointByID(id: int): Waypoint: Visszaadja az id-hez tartozó Waypoint-ot.
		\item canBuildObstacle(pos: Vector): boolean: Visszaadja, hogy a pos helyre építhető-e akadály.
		\item canBuildTower(pos: Vector): boolean: Visszaadja, hogy a pos helyre építhető-e torony.
	\end{itemize}
\end{itemize}

\subsection{Mission}
\begin{itemize}
\item Felelősség\newline
%\comment{Mi az osztály felelőssége. Kb 1 bekezdés. Ha szükséges, akkor state-chart is.}
Betölt egy XML küldetésleíró fájlt, és létrehoz az összes ellenséghez egy Enemy objektumot, amiket eltárol, hogy később le lehessen kérni tőle a következő ellenséget.
\item Attribútumok\newline
%\comment{Milyen attribútumai vannak}
	\begin{itemize}
		\item -spawnList: List<Spawn>: Az Enemy-ket és a hozzájuk tartozó spawn időket tárolja.
		\item name: String: A misszió neve.
	\end{itemize}
\item Metódusok\newline
%\comment{Milyen publikus, protected és privát  metódusokkal rendelkezik. Metódusonként precíz leírás, ha szükséges, activity diagram is  a metódusban megvalósítandó algoritmusról.}
	\begin{itemize}
		\item Mission(file: String, map: Map): Létrehoz egy Mission osztályt a paraméterként megadott küldetésleíró fájlból.
		\item getNextEnemy(): Enemy: Visszaadja a következő ellenséget amit el kell indítani a pályán, vagy null.
	\end{itemize}
\end{itemize}

\subsection{Tower}
\begin{itemize}
\item Felelősség\\
Felelős \textbf{Projectile}-ok létrehozásához, azok megfelelő felparaméterezésével. Továbbá felelős azért, hogy \textbf{Projectile}-okat csak a megadott időközönként lőjjön ki.
\item Attribútumok
	\begin{itemize}
		\item \underline{cost: int}: A torony ára varázserőben.
		\item  \underline{range: double}: A távolság amire tud lőni.
		\item \underline{fireRate: double}: A lövési gyakoriság.
		\item - cooldown: double: Mennyi idő van még a következő lövésig.
		\item - \underline{critical: boolean}: Kritikusat sebez-e.
		\item - gem: \textbf{TowerGem}: Eltárol egy referenciát egy \textbf{Gem} típusú objektumra, ami meghatározza, hogy az adott épület milyen echant alatt áll.
		\item - damage: \textbf{HashMap}<\textbf{EnemyType}, double>: Megadja mekkora az adott típusú ellenfélre kifejtett hatása a toronynak.
		\item - position: Vector: Visszatér az épület koordinátáival.
	\end{itemize}
\item Metódusok
	\begin{itemize}
		\item + attack(List <\textbf{Enemy}>): \textbf{Projectile}: Először megnézi, hogy lőhet-e, ha nem akkor semmivel se tér vissza. Ha igen akkor végignézi a kapott listában az ellenségeket, és amelyik a hatótávolságán belül van, és a legközelebb a célhoz, arra kilő egy \textbf{Projectile}-t, majd a visszatérési értékében visszaadja azt. A \textbf{Projectile}-t felparaméterezi az ellenséghez megfelelő sebzési adatokkal.
		\item + doesCollide(pos: Vector): Visszatér azzal, hogy a Vector ütközik-e a toronnyal.
		\item + getCost(): int: Visszatér a cost attribútummal.
		\item + getGem(): \textbf{Gem}: Visszaadja az épületen található varázskövet.
		\item + setGem(\textbf{TowerGem} gem): void: Beállítja az epületen lévő varázskövet. Ha már volt az épületen varázskő, akkor az előző megszűnik.
		\item + getPosition(): Vector: Visszaadja a position attribútumot.
		\item + getRange(): double: Visszatér a range-el. 
	\end{itemize}
\end{itemize}


\subsection{Obstacle}
\begin{itemize}
\item Felelősség\\
Felelős, az ellenfelek lassításáért, úgy, hogy meg kell tudnia mondani a pozícióját, valamint, hogy az adott ellenséget mennyire lassítja.
\item Attribútumok
	\begin{itemize}
		\item \underline{cost}: int: Az akadály ára varázserőben.
		\item - gem: \textbf{ObstacleGem}: Eltárol egy referenciát egy \textbf{Gem} típusú objektumra, ami meghatározza, hogy az adott épület milyen echant alatt áll.
		\item - slowingFactor: \textbf{Map}<\textbf{EnemyType}, double>: Megadja mekkora az adott típusú ellenfélre kifejtett hatása az akadálynak.
		\item - position: \textbf{Vector}: Az akadály koordinátáit tárolja.
		\item - range: double: Az akadály hatótávolsága.
	\end{itemize}
\item Metódusok
	\begin{itemize}
		\item + Obstacle(pos: Vector): Létrehoz egy \textbf{Obstacle} objektumot, a pozícióját pos-ra állítva.
		\item + doesCollide(pos: Vector): boolean: Visszatér azzal, hogy ütközik-e a pos az akadállyal.
		\item + getCost(): int: Visszatér a torony árával.
		\item + getSlowingFactor(\textbf{Enemy} enemy): double: Visszatér azzal az értékkel, amivel az adott ellenfelet lassítja.
		\item + getGem(): \textbf{Gem}: Visszaadja az épületen található varázskövet.
		\item + setGem(\textbf{Gem} gem): void: Beállítja az epületen lévő varázskövet. Ha már volt az épületen varázskő, akkor az előző megszűnik.
		\item + getPosition(): Vector: Visszaadja a position attribútumot.
		\item + getRange(): double: Visszaadja az akadály hatótávolságát.
	\end{itemize}
\end{itemize}


\subsection{ObstacleGem}
\begin{itemize}
\item Felelősség\\
Egy akadályra rakható varázskő tulajdonságainak tárolása.
\item Ősosztályok\\
Gem
\item Attribútumok
	\begin{itemize}
		\item \underline{yellow: ObstacleGem}: Sárga ObstacleGem.
		\item \underline{orange: ObstacleGem}: Narancssárga ObstacleGem.
		\item - speed: HashMap<\textbf{EnemyType}, Double>: Megadja, hogy a varázskővel elátott akadályon áthaladó adott típusú ellenség sebessége hányadára csökken.
	\end{itemize}
\item Metódusok
	\begin{itemize}
		\item + getSpeedMultiplier(\textbf{EnemyType} enemyType): double: Visszaadja varázskő sebesség szorzóját egy adott típusú ellenséghez.
	\end{itemize}
\end{itemize}

\subsection{TowerGem}
\begin{itemize}
\item Felelősség\\
Egy toronyra rakható varázskő tulajdonságainak tárolása.
\item Ősosztályok\\
Gem
\item Attribútumok
	\begin{itemize}
		\item \underline{red: TowerGem}: Piros TowerGem.
		\item \underline{green: TowerGem}: Zöld TowerGem.
		\item \underline{blue: TowerGem}: Kék TowerGem.
		\item - rate: double: Megadja, hogy a varázskővel ellátott toronynak hányszorosára nő a tüzelési sebessége.
		\item - damage: HashMap<\textbf{EnemyType}, double>: Megadja, hogy a varázskővel ellátott toronynak hányszorosára nő a sebzése egy adott típusú ellenséggel szemben.
	\end{itemize}
\item Metódusok
	\begin{itemize}
		\item + getRateMultiplier(): double: Visszaadja a varázskő tüzelési sebesség szorzójáz.
		\item + getDamageMultiplier(\textbf{EnemyType} enemyType): double: Visszaadja varázskő sebzés szorzóját egy adott típusú ellenséghez.
	\end{itemize}
\end{itemize}

\section{A tesztek részletes tervei, leírásuk a teszt nyelvén}
%[A tesztek részletes tervei alatt meg kell adni azokat a bemeneti adatsorozatokat, amelyekkel a program működése ellenőrizhető. Minden bemenő adatsorozathoz definiálni kell, hogy az adatsorozat végrehajtásától a program mely részeinek, funkcióinak ellenőrzését várjuk és konkrétan milyen eredményekre számítunk, ezek az eredmények hogyan vethetők össze a bemenetekkel.]

\subsection{Alapvető működés}
\begin{itemize}
\item Leírás\newline
Ennek a tesztnek az alapegységek működésének tesztelése a célja.
\item Ellenőrzött funkcionalitás, várható hibahelyek\newline
Egy pálya és egy misszió betöltése, egy torony és egy akadály építése, egy időegység léptetése, valamint a tornyok, lövedékek, és ellenségek listázása, ezután pedig az épületek megerősítése, és az ellenség útválasztása kerül ellenőrzésre. Lehetséges hibák: A torony vagy az akadály nem épül meg. Az ellenség nem követi a számára kijelölt utat. Az épületek erősítése nem történik meg megfelelően. A torony nem sebzi a mellette elhaladó ellenséget, vagy létre sem jön lövedék.
\item Bemenet\newline
loadMap basic\_test\_map\newline
loadMission basic\_test\_mission\newline
buildTower 2 5\newline
buildObstacle 10 4\newline
step 1\newline
listTowers\newline
listObstacles\newline
listProjectiles\newline
listEnemies\newline
enchant 1 2 5\newline
enchant 2 10 4\newline
setWaypoint 3\newline
step 100\newline
listTowers\newline
listObstacles\newline
listEnemies\newline
\item Elvárt kimenet\newline
1 (2;5)	-\newline
1 (10;4)	-\newline
1 (2;5)	1	false\newline
1 100	(0;4)\newline
1 (2;5)	1\newline
1 (10;4)	2\newline
1 60	(20;6)\newline
\end{itemize}


\subsection{Köd ellenőrzése}
\begin{itemize}
\item Leírás\newline
Ennek a tesztnek a köd működésének tesztelése a célja.
\item Ellenőrzött funkcionalitás, várható hibahelyek\newline
A ködben létrejövő látótávolság-csökkenés hatását vizsgálja. Lehetséges hiba, hogy a köd nem csökkenti megfelelő mértékben a látótávolságot, így az az ellenség is sebződik, amelyik túl messze van a toronytól.
\item Bemenet\newline
loadMap fog\_test\_map\newline
loadMission fog\_test\_mission\newline
buildTower 2 5\newline
setFog 1\newline
step 100\newline
listEnemies\newline
\item Elvárt kimenet\newline
1 100	(20;10)\newline
\end{itemize}


\subsection{Erős lövés ellenőrzése}
\begin{itemize}
\item Leírás\newline
Ennek a tesztnek az erős lövedék hatásának tesztelése a célja.
\item Ellenőrzött funkcionalitás, várható hibahelyek\newline
A tornyokban elvétve előforduló erős lövés hatását teszteli. Lehetséges hiba, hogy az ellenség nem osztódik ketté.
\item Bemenet\newline
loadMap fog\_test\_map\newline
loadMission fog\_test\_mission\newline
buildTower 2 5\newline
setCritical 1
step 100\newline
listEnemies\newline
\item Elvárt kimenet\newline
1 40	(20;10)\newline
2 40	(20;10)\newline
\end{itemize}


\section{A tesztelést támogató programok tervei}
%\comment{A tesztadatok előállítására, a tesztek eredményeinek kiértékelésére szolgáló segédprogramok részletes terveit kell elkészíteni.}

A tesztelést végző segédprogram a specifikált formátumban megírt tesztfájlokat olvassa be, elindítja az alkalmazást, megadja neki az előírt bemenetet, majd a kapott kimenetet összehasonlítja a megaditt elvárt kimenettel, és amelyik sorban eltérést észlel, ott jelzi a különbséget. Ha nincs eltérés, a teszt sikeresen lefutott.