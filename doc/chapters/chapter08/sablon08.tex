% Szglab4
% ===========================================================================
%
\chapter{Részletes tervek}

\thispagestyle{fancy}

\section{Osztályok és metódusok tervei}

\subsection{Waypoint}
\begin{itemize}
\item Felelősség\\
Útvonalat kijelölni az ellenségeknek, úgy, hogy megadja a pozícióját, amely felé az ellenségek mehetnek, valamint a következő  \textbf{Waypoint}-ot ami felé menniük kell, ha egyszer elérték ezt a  \textbf{Waypoint}-ot.
\item Attribútumok
	\begin{itemize}
		\item - position: Vector: A \textbf{Waypoint} pozíciója a pályán.
		\item - distance: double: A  \textbf{Waypoint} távolságát a céltól tárolja.
		\item - nextWaypoints: List <Waypoint, double>: A következő  \textbf{Waypoint}-okat és a hozzájuk tartozó valószínűségeket
	\end{itemize}
\item Metódusok
	\begin{itemize}
		\item + getDistance():double: Visszaadja a distance attribútumot.
		\item + getNextWaypoint(): \textbf{Waypoint}: Visszatér a nextWaypoints listából véletlenszerűen kiválasztott \textbf{Waypoint}-al
		\item + getPosition(): Vector: visszatér a position attribútummal.
		\item + \textbf{Waypoint}(Vector v): Konstruktor, beállítja a position attribútumot. 
	\end{itemize}
\end{itemize}

\subsection{Tower}
\begin{itemize}
\item Felelősség\\
Felelős \textbf{Projectile}-ok létrehozásához, azok megfelelő felparaméterezésével. Továbbá felelős azért, hogy \textbf{Projectile}-okat csak a megadott időközönként lőjjön ki.
\item Attribútumok
	\begin{itemize}
		\item \underline{cost}: int: A torony ára varázserőben.
		\item - range: double: A távolság amire tud lőni.
		\item - cooldown: double: A lövések gyakorisága.
		\item - gem: \textbf{Gem}: Eltárol egy referenciát egy \textbf{Gem} típusú objektumra, ami meghatározza, hogy az adott épület milyen echant alatt áll.
		\item - double: \textbf{HashMap}<\textbf{EnemyType}, double>: Megadja mekkora az adott típusú ellenfélre kifejtett hatása a toronynak.
		\item - position: Vector: Visszatér az épület koordinátáival.
	\end{itemize}
\item Metódusok
	\begin{itemize}
		\item + attack(List <\textbf{Enemy}>): \textbf{Projectile}: Először megnézi, hogy lőhet-e, ha nem akkor semmivel se tér vissza. Ha igen akkor végignézi a kapott listában az ellenségeket, és amelyik a hatótávolságán belül van, és a legközelebb a célhoz, arra kilő egy \textbf{Projectile}-t, majd a visszatérési értékében visszaadja azt. A \textbf{Projectile}-t felparaméterezi az ellenséghez megfelelő sebzési adatokkal.
		\item + getCost(): int: Visszatér a cost attribútummal.
		\item + getGem(): \textbf{Gem}: Visszaadja az épületen található varázskövet.
		\item + setGem(\textbf{TowerGem} gem): void: Beállítja az epületen lévő varázskövet. Ha már volt az épületen varázskő, akkor az előző megszűnik.
		\item + getPosition(): Vector: Visszaadja a position attribútumot.
	\end{itemize}
\end{itemize}


\subsection{Obstacle}
\begin{itemize}
\item Felelősség\\
Felelős, az ellenfelek lassításáért, úgy, hogy meg kell tudnia mondani a pozícióját, valamint, hogy az adott ellenséget mennyire lassítja.
\item Attribútumok
	\begin{itemize}
		\item \underline{cost}: int: Az akadály ára varázserőben.
		\item - gem: \textbf{Gem}: Eltárol egy referenciát egy \textbf{Gem} típusú objektumra, ami meghatározza, hogy az adott épület milyen echant alatt áll.
		\item - slowingFactor: \textbf{Map}<\textbf{EnemyType}, double>: Megadja mekkora az adott típusú ellenfélre kifejtett hatása az akadálynak.
		\item - position: \textbf{Vector}: Az akadály koordinátáit tárolja.
		\item - range: double: Az akadály hatótávolsága.
	\end{itemize}
\item Metódusok
	\begin{itemize}
		\item + Obstacle(Vector pos): Létrehoz egy \textbf{Obstacle} objektumot, a pozícióját pos-ra állítva.
		\item + getCost(): int: Visszatér a torony árával.
		\item + getSlowingFactor(\textbf{Enemy} enemy): double: Visszatér azzal az értékkel, amivel az adott ellenfelet lassítja.
		\item + getGem(): \textbf{Gem}: Visszaadja az épületen található varázskövet.
		\item + setGem(\textbf{Gem} gem): void: Beállítja az epületen lévő varázskövet. Ha már volt az épületen varázskő, akkor az előző megszűnik.
		\item + getPosition(): Vector: Visszaadja a position attribútumot.
		\item + getRange(): double: Visszaadja az akadály hatótávolságát.
	\end{itemize}
\end{itemize}


\subsection{ObstacleGem}
\begin{itemize}
\item Felelősség\\
Egy akadályra rakható varázskő tulajdonságainak tárolása.
\item Ősosztályok\\
Gem
\item Attribútumok
	\begin{itemize}
		\item - speedMultiplier: HashMap<\textbf{EnemyType}, Double>: Megadja, hogy a varázskővel elátott akadályon áthaladó adott típusú ellenség sebessége hányadára csökken.
	\end{itemize}
\item Metódusok
	\begin{itemize}
		\item + getSpeedMultiplier(\textbf{EnemyType} enemyType): double: Visszaadja varázskő sebesség szorzóját egy adott típusú ellenséghez.
	\end{itemize}
\end{itemize}

\subsection{TowerGem}
\begin{itemize}
\item Felelősség\\
Egy toronyra rakható varázskő tulajdonságainak tárolása.
\item Ősosztályok\\
Gem
\item Attribútumok
	\begin{itemize}
		\item - rateMultiplier: double: Megadja, hogy a varázskővel ellátott toronynak hányszorosára nő a tüzelési sebessége.
		\item - damageMultiplier: HashMap<\textbf{EnemyType}, double>: Megadja, hogy a varázskővel ellátott toronynak hányszorosára nő a sebzése egy adott típusú ellenséggel szemben.
	\end{itemize}
\item Metódusok
	\begin{itemize}
		\item + getRateMultiplier(): double: Visszaadja a varázskő tüzelési sebesség szorzójáz.
		\item + getDamageMultiplier(\textbf{EnemyType} enemyType): double: Visszaadja varázskő sebzés szorzóját egy adott típusú ellenséghez.
	\end{itemize}
\end{itemize}


\section{A tesztek részletes tervei, leírásuk a teszt nyelvén}
[A tesztek részletes tervei alatt meg kell adni azokat a bemeneti adatsorozatokat, amelyekkel a program működése ellenőrizhető. Minden bemenő adatsorozathoz definiálni kell, hogy az adatsorozat végrehajtásától a program mely részeinek, funkcióinak ellenőrzését várjuk és konkrétan milyen eredményekre számítunk, ezek az eredmények hogyan vethetők össze a bemenetekkel.]

\subsection{Teszteset1}
\begin{itemize}
\item Leírás\newline
\comment{szöveges leírás, kb. 1-5 mondat.}
\item Ellenőrzött funkcionalitás, várható hibahelyek
\item Bemenet\newline
\comment{a proto bemeneti nyelvén megadva (lásd előző anyag)}
\item Elvárt kimenet\newline
\comment{a proto kimeneti nyelvén megadva (lásd előző anyag)}
\end{itemize}

\subsection{Teszteset2}
\begin{itemize}
\item Leírás\newline
\comment{szöveges leírás, kb. 1-5 mondat.}
\item Ellenőrzött funkcionalitás, várható hibahelyek
\item Bemenet\newline
\comment{a proto bemeneti nyelvén megadva (lásd előző anyag)}
\item Elvárt kimenet\newline
\comment{a proto kimeneti nyelvén megadva (lásd előző anyag)}
\end{itemize}

\section{A tesztelést támogató programok tervei}
\comment{A tesztadatok előállítására, a tesztek eredményeinek kiértékelésére szolgáló segédprogramok részletes terveit kell elkészíteni.}

