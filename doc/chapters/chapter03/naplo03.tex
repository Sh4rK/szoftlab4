% Szglab4
% ===========================================================================
%
\section{Napló}

\begin{naplo}

\bejegyzes
{2014.02.26.~8:15~} % Kezdet
{1,5 óra} % Időtartam
{\vadam\newline
\vantal\newline
\vbator\newline
\vtorok}
{Elkezdtük tárgyalni a feladatot. Megvitattuk a program alapvető működését, a főbb osztályokat, metódusokat, szekvenciákat.}

\bejegyzes
{2014.02.27.~14:15~}
{1 óra}
{\vadam\newline
\vantal\newline
\vbator\newline
\vtorok}
{Folytattuk a szerdai megbeszélést. Kiosztottuk a feladatokat. \newline Felelősségek: \newline
Ádám: Map, Tile, Mission \newline
Antal: Gem, Projectile \newline
Bátor: Building, Obstacle, Tower \newline
Török: Enemy, EnemyType, Game \newline osztályok}

\bejegyzes
{2014.02.28.~16:30~}
{2,5 óra}
{\vadam}
{Map, Tile és Mission osztályok leírása.}

\bejegyzes
{2014.02.28.~18:30~}
{1,5 óra}
{\vbator}
{Tower, Obstacle osztályok leírása}

\bejegyzes
{2014.02.28.~19:00~}
{1 óra}
{\vantal}
{Projectile osztály leírása.}

\bejegyzes
{2014.02.28.~20:00~}
{1 óra}
{\vtorok}
{Game, EnemyType, Enemy osztályok leírása.}

\bejegyzes
{2014.03.02.~20:00~}
{2 óra}
{\vadam}
{Szekvenciadiagrammok, állapotdiagrammok készítése. }

\bejegyzes
{2014.03.02.~20:00~}
{4 óra}
{\vbator}
{Szekvenciadiagrammok, osztálydiagram készítése. }

\bejegyzes
{2014.03.01.~18:30~}
{1.5 óra}
{\vadam}
{Szekvencia diagramok készítése.}

\bejegyzes
{2014.03.02.~20:00~}
{2 óra}
{\vadam}
{State chart diagramok készítése.}

\end{naplo}

