% Szglab4
% ===========================================================================
%
\chapter{Analízis modell kidolgozása 1}

\thispagestyle{fancy}

\section{Objektum katalógus}

\comment{Minden, a feladatban szereplő objektum rövid, egy-két bekezdés hosszú ismertetése. Meg kell jelenjen minden objektumhoz, hogy mi a felelőssége. Informális leírás, ezért nem kell foglalkozni az örökléssel, az interfészekkel, az absztrakt osztályokkal, a segédosztályokkal.}

\subsection{Tower}
Ha egy ellenség a hatókörébe ér, akkor elindít egy \textbf{Projectile}-t az irányába, a tüzelési gyakoriságának megfelelő időközönként. Ha több ellenfél van a hatókörében, akkor arra lő, aki a legközelebb van a célhoz.

\subsection{Obstacle}
Ha egy ellenség belelép egy olyan mezőbe, amin van egy \textbf{Obstacle}, akkor azt egy bizonyos mértékben lelassítja, az ellenség fajától függően.

\section{Statikus struktúra diagramok}
\comment{Az előző alfejezet osztályainak kapcsolatait és publikus metódusait bemutató osztálydiagram(ok). Tipikus hibalehetőségek: csillag-topológia, szigetek.}

\begin{figure}[h]
\begin{center}
%\includegraphics[width=17cm]{chapters/chapter03/example.pdf}
\caption{x}
\label{fig:example1}
\end{center}
\end{figure}


\section{Osztályok leírása}
\comment{Az előző alfejezetben tárgyalt objektumok felelősségének formalizálása attribútumokká, metódusokká. Csak publikus metódusok szerepelhetnek. Ebben az alfejezetben megjelennek az interfészek, az öröklés, az absztrakt osztályok. Segédosztályokra még mindig nincs szükség. Az osztályok ABC sorrendben kövessék egymást. Interfészek esetén az Interfészek, Attribútumok pontok kimaradnak.}

\subsection{Building}
\begin{itemize}
\item Felelősség\\
%\comment{Mi az osztály felelőssége. Kb 1 bekezdés.} \\
Biztosít egy interfészt az épületek számára. Eltárolja az épülethez tartozó varázskövet.
\item Attribútumok
%\\\comment{Milyen attribútumai vannak}
	\begin{itemize}
		\item gem: eltárol egy referenciát egy \textbf{Gem} típusú objektumra, ami meghatározza, hogy az adott épület milyen echant alatt áll.
		\item int human: megadja mekkora a human típusú ellenfelekre kifejtett hatása az épületnek.
		\item int elf: megadja mekkora az elf típusú ellenfelekre kifejtett hatása az épületnek.
		\item int dwarf: megadja mekkora a dwarf típusú ellenfelekre kifejtett hatása az épületnek.
		\item int hobbit: megadja mekkora a hobbit típusú ellenfelekre kifejtett hatása az épületnek.
		\item \textbf{Tile} tile: egy referencia arra a mezőre, ahol az épület található.
	\end{itemize}
\item Metódusok\\
%\comment{Milyen publikus metódusokkal rendelkezik. Metódusonként egy-három mondat arról, hogy a metódus mit csinál.}
	\begin{itemize}
		\item \textbf{Gem} getGem(): visszaadja az épületen található varázskövet
		\item void setGem(\textbf{Gem} gem): beállítja az epületen lévő varázskövet. Ha már volt az épületen varázskő, akkor az előző megszűnik.
		\item void setTile(\textbf{Tile} tile)
		\item \textbf{Tile} getTile()
	\end{itemize}
\end{itemize}

\subsection{Tower}
\begin{itemize}
\item Felelősség\\
Felelős \textbf{Projectile}-ok létrehozásához, azok megfelelő felparaméterezésével. Továbbá felelős azért, hogy \textbf{Projectile}-okat csak a megadott időközönként lőjjön ki.
\item Ősosztályok\\
\textbf{Building}
\item Attribútumok
	\begin{itemize}
		\item interval: milyen időközönként tud a torony lőni.
		\item timeSinceLastFire: mennyi idő telt el az utolsó lövés óta.
	\end{itemize}
\item Metódusok
	\begin{itemize}
		\item \textbf{Projectile} attack(\textbf{Enemy} enemy): kilő egy \textbf{Projectile}-t a paraméterként megkapott ellenségre, majd a visszatérési értékében visszaadja azt. A \textbf{Projectile}-t felparaméterezi az ellenséghez megfelelő sebzési adatokkal.
	\end{itemize}
\end{itemize}

\subsection{Obstacle}
\begin{itemize}
\item Felelősség\\
Felelős \textbf{Projectile}-ok létrehozásához, azok megfelelő felparaméterezésével. Továbbá felelős azért, hogy \textbf{Projectile}-okat csak a megadott időközönként lőjjön ki.
\item Ősosztályok\\
\textbf{Building}
\item Attribútumok\\
		Nincsenek attribútumai.
\item Metódusok
	\begin{itemize}
		\item int getSlowingFactor(\textbf{Enemy} enemy): visszatér azzal az értékkel, amivel az adott ellenfelet lassítja.
	\end{itemize}
\end{itemize}

\section{Statikus struktúra diagramok}
\comment{Az előző alfejezet osztályainak kapcsolatait és publikus metódusait bemutató osztálydiagram(ok). Tipikus hibalehetőségek: csillag-topológia, szigetek.}

\begin{figure}[h]
\begin{center}
%\includegraphics[width=17cm]{chapters/chapter03/example.pdf}
\caption{x}
\label{fig:example1}
\end{center}
\end{figure}

\section{Szekvencia diagramok}
\comment{Inicializálásra, use-case-ekre, belső működésre. Konzisztens kell legyen az előző alfejezettel. Minden metódus, ami ott szerepel, fel kell tűnjön valamelyik szekvenciában. Minden metódusnak, ami szekvenciában szerepel, szereplnie kell a valamelyik osztálydiagramon.}

\begin{figure}[h]
\begin{center}
%\includegraphics[width=17cm]{chapters/chapter03/example.pdf}
\caption{x}
\label{fig:example2}
\end{center}
\end{figure}

\section{State-chartok}
\comment{Csak azokhoz az osztályokhoz, ahol van értelme. Egyetlen állapotból álló state-chartok ne szerepeljenek. A játék működését bemutató state-chart-ot készíteni tilos.}

\begin{figure}[h]
\begin{center}
%\includegraphics[width=17cm]{chapters/chapter03/example.pdf}
\caption{x}
\label{fig:example3}
\end{center}
\end{figure}

