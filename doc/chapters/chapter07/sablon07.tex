% Szglab4
% ===========================================================================
%
\chapter{Prototípus koncepciója}

\thispagestyle{fancy}



\section{Prototípus interface-definíciója}
%\comment{Definiálni kell a teszteket leíró nyelvet. Külön figyelmet kell fordítani arra, hogy ha a rendszer véletlen elemeket is tartalmaz, akkor a véletlenszerűség ki-bekapcsolható legyen, és a program determinisztikusan is tesztelhető legyen.}

\subsection{Az interfész általános leírása}
%\comment{A protó (karakteres) input és output felületeit úgy kell kialakítani, hogy az input fájlból is vehető legyen illetőleg az output fájlba menthető legyen, vagyis kommunikációra csak a szabványos be- és kimenet használható.}

Az interfész csak a szabványos bemenetről fogad parancsokat, és a szabványos kimenetre írja ki az esetleges kimenetet. Ezáltal terminálból is használható, valamint az elkészítendő tesztelő segédprogram segítségével átirányítható a ki- és bemenet fájlokra, így van mód automatikus tesztelésre, előre elkészített teszteseteket felhasználva. A tesztesetek a prototípusnak adandó parancsok sorozatából, valamint az adott sorozatra adandó helyes kimenet található. A tesztek sikeresek, ha a valós, és a leírt elvárt kimenet megegyezik.

\subsection{Bemeneti nyelv}
%\comment{Definiálni kell a teszteket leíró nyelvet. Külön figyelmet kell fordítani arra, hogy ha a rendszer véletlen elemeket is tartalmaz, akkor a véletlenszerűség ki-bekapcsolható legyen, és a program determinisztikusan is futtatható legyen. A szálkezelést is tesztelhető, irányítható módon kell megoldani.}

A prototípus által elfogadott parancsok a következők:

\begin{itemize}

\item loadMap
	\begin{itemize}
	\item Leírás: Egy pálya betöltése
	\item Opciók: A betöltendő pálya neve
	\end{itemize}

\item loadMission
	\begin{itemize}
	\item Leírás: Egy misszió betöltése
	\item Opciók: A betöltendő misszió neve
	\end{itemize}

\item setFog
	\begin{itemize}
	\item Leírás: A köd beállítása
	\item Opciók: A köd kívánt állapota: 0 - nincs köd, 1 - van köd
	\end{itemize}

\item setCritical
	\begin{itemize}
	\item Leírás: A kettévágó lövedék beállítása
	\item Opciók: A kívánt hatás: 0 - nem kettévágó lövedék, 1 - kettévágó lövedék
	\end{itemize}

\item setWaypoint
	\begin{itemize}
	\item Leírás: A csomópontokból való továbbhaladás irányának beállítása
	\item Opciók: A következő csomópont azonosítója
	\end{itemize}

\item step
	\begin{itemize}
	\item Leírás: A játéklogika léptetése
	\item Opciók: A léptetni kívánt időegységek száma
	\end{itemize}

\item buildTower
	\begin{itemize}
	\item Leírás: Toronyépítés
	\item Opciók: Az építendő torony koordinátái
	\end{itemize}

\item buildObstacle
	\begin{itemize}
	\item Leírás: Akadály építése
	\item Opciók: Az építendő akadály koordinátái
	\end{itemize}

\item enchant
	\begin{itemize}
	\item Leírás: Torony vagy akadály felszerelése drágakővel
	\item Opciók: A drágakő típusa, illetve a felszerelendő épület koordinátái
	\end{itemize}

\item listEnemies
	\begin{itemize}
	\item Leírás: A pályán lévő ellenségek kilistázása
	\item Opciók: -
	\end{itemize}

\item listTowers
	\begin{itemize}
	\item Leírás: A tornyok kilistázása
	\item Opciók: -
	\end{itemize}

\item listObstacles
	\begin{itemize}
	\item Leírás: Az akadályok kilistázása
	\item Opciók: -
	\end{itemize}

\item listProjectiles
	\begin{itemize}
	\item Leírás: A lövedékek kilistázása
	\item Opciók: -
	\end{itemize}


\end{itemize}

%\comment{Ha szükséges, meg kell adni a konfigurációs (pl. pályaképet megadó) fájlok nyelvtanát is.}

A pályákat és missziókat leíró fájlok XML formátumban lesznek, a következő DTD-k szerint:\newline
\newline
<!DOCTYPE map\newline
[\newline
<!ELEMENT map (name, waypoint*, route)>\newline
<!ELEMENT name (\#PCDATA)\newline
<!ELEMENT waypoint (id, coords)>\newline
<!ELEMENT id (\#PCDATA)\newline
<!ELEMENT coords (x, y)>\newline
<!ELEMENT x (\#PCDATA)\newline
<!ELEMENT y (\#PCDATA)\newline
<!ELEMENT route (a, b, chance)>\newline
<!ELEMENT a (\#PCDATA)\newline
<!ELEMENT b (\#PCDATA)\newline
<!ELEMENT chance (\#PCDATA)\newline
]>\newline
\newline
<!DOCTYPE mission\newline
[\newline
<!ELEMENT mission (name, enemy*)>\newline
<!ELEMENT name (\#PCDATA)\newline
<!ELEMENT enemy (id, type)>\newline
<!ELEMENT id (\#PCDATA)\newline
<!ELEMENT type (\#PCDATA)\newline
]>\newline

\subsection{Kimeneti nyelv}
%\comment{Egyértelműen definiálni kell, hogy az egyes bemeneti parancsok végrehajtása után előálló állapot milyen formában jelenik meg a szabványos kimeneten.}

Kimenetet csak a következő 4 parancs ad, a következő formában:\newline
\newline
listEnemies\newline
Az összes élő ellenséget írja ki, soronként egyet:\newline
<sorszám> <életerő> <koordináták>\newline
\newline
listTowers\newline
Az összes tornyot írja ki, soronként egyet:\newline
<sorszám> <koordináták> <varázskő>\newline
\newline
listObstacles\newline
Az összes tornyot írja ki, soronként egyet:\newline
<sorszám> <koordináták> <varázskő>\newline
\newline
listProjectiles\newline
Az összes tornyot írja ki, soronként egyet:\newline
<sorszám> <koordináták> <célpont> <szétvágó-e>\newline
\newline
\section{Összes részletes use-case}
\comment{A use-case-eknek a részletezettsége feleljen meg a kezelői felületnek, azaz a felület elemeire kell hivatkozniuk.
Alábbi táblázat minden use-case-hez külön-külön.}

\begin{figure}[h]
\begin{center}
%\includegraphics[width=17cm]{chapters/chapter07/example.pdf}
\caption{x}
\label{fig:ProtoUseCase}
\end{center}
\end{figure}

\usecase{...}{...}{...}{...}

\section{Tesztelési terv}
\comment{A tesztelési tervben definiálni kell, hogy a be- és kimeneti fájlok egybevetésével miként végezhető el a program tesztelése. Meg kell adni teszt forgatókönyveket. Az egyes teszteket elég informálisan, szabad szövegként leírni. Teszt-esetenként egy-öt mondatban. Minden teszthez meg kell adni, hogy mi a célja, a proto mely funkcionalitását, osztályait stb. teszteli. Az alábbi táblázat minden teszt-esethez külön-külön elkészítendő.}

\teszteset{...}{...}{...}

\section{Tesztelést támogató segéd- és fordítóprogramok specifikálása}
\comment{Specifikálni kell a tesztelést támogató segédprogramokat.}

