% Szglab4
% ===========================================================================
%
\chapter{Prototípus beadása}

\thispagestyle{fancy}

\section{Fordítási és futtatási útmutató}
% \comment{A feltöltött program fordításával és futtatásával kapcsolatos útmutatás. Ennek tartalmaznia kell leltárszerűen az egyes fájlok pontos nevét, méretét byte-ban, keletkezési idejét, valamint azt, hogy a fájlban mi került megvalósításra.}

\subsection{Fájllista}

\begin{fajllista}

\fajl
{Enemy.java}
{3 473 byte}
{2014.03.31~10:16~}
{Az ellenségeket megvalósító osztály.}

\fajl
{EnemyType.java}
{1 042 byte}
{2014.03.31~10:16~}
{Az ellenségek típusait leíró osztály.}

\fajl
{Fog.java}
{513 byte}
{2014.04.21~11:46~}
{A köd viselkedését meghatározó osztály.}

\fajl
{Game.java}
{11 727 byte}
{2014.03.31~10:16~}
{A játék mechanikáját tartalmazó osztály.}

\fajl
{Gem.java}
{357 byte}
{2014.03.31~10:16~}
{A varázskövek közös absztrakt ősosztálya.}

\fajl
{Map.java}
{3 628 byte}
{2014.03.31~10:16~}
{Egy térképet tároló osztály.}

\fajl
{Mission.java}
{2 390 byte}
{2014.03.31~10:16~}
{Egy küldetés menetét leíró osztály.}

\fajl
{Obstacle.java}
{2 116 byte}
{2014.03.31~10:16~}
{Egy akadályt megvalósító osztály.}

\fajl
{ObstacleGem.java}
{821 byte}
{2014.03.31~10:16~}
{Az akadályokra rakható varázskő osztálya.}

\fajl
{Pair.java}
{225 byte}
{2014.04.21~13:24~}
{Egy egyszerű generikus pár.}

\fajl
{Projectile.java}
{1 228 byte}
{2014.03.31~10:16~}
{A lövedékeket működtető osztály.}

\fajl
{Spawn.java}
{231 byte}
{2014.04.22~6:59~}
{Egy ellenség megjelenését tároló osztály.}

\fajl
{SplitterProjectile.java}
{760 byte}
{2014.04.21~15:42~}
{Speciális, az ellenséget kettévágó Projectile.}

\fajl
{Tower.java}
{3 696 byte}
{2014.03.31~10:16~}
{Egy tornyot leíró osztály.}

\fajl
{TowerGem.java}
{928 byte}
{2014.03.31~10:16~}
{A tornyokat erősítő varázsköveket tárolja.}

\fajl
{Vector.java}
{1 356 byte}
{2014.03.31~10:16~}
{Egy két dimenziós valós vektor.}

\fajl
{Waypoint.java}
{2 319 byte}
{2014.03.31~10:16~}
{Az ellenségek útvonalainak egy csomópontja.}


\fajl
{Main.java}
{4 548 byte}
{2014.04.22~11:09~}
{A teszter segédprogram főosztálya.}


\end{fajllista}

\subsection{Fordítás}
%\comment{A fenti listában szereplő forrásfájlokból milyen műveletekkel lehet a bináris, futtatható kódot előállítani. Az előállításhoz csak a 2. Követelmények c. dokumentumban leírt környezetet szabad előírni.}

\lstset{escapeinside=`', xleftmargin=10pt, frame=single, basicstyle=\ttfamily\footnotesize, language=sh}
\begin{lstlisting}
javac -encoding utf8 -d bin src/szoftlab4/*.java src/Tester/*.java
\end{lstlisting}

\subsection{Futtatás}
%\comment{A futtatható kód elindításával kapcsolatos teendők leírása. Az indításhoz csak a 2. Követelmények c. dokumentumban leírt környezetet szabad előírni.}

\lstset{escapeinside=`', xleftmargin=10pt, frame=single, basicstyle=\ttfamily\footnotesize, language=sh}
Teszter indítása
\begin{lstlisting}
cd tests
java -cp ../bin Tester.Main
\end{lstlisting}

Program indítása, ekkor a standard bementetről lehet parancsokat adni a programnak
\begin{lstlisting}
cd bin
java szoftlab4.Main
\end{lstlisting}

Ilyenkor lehetséges előre elkészített teszteseteket is lefuttatni, pl így:
\begin{lstlisting}
cd tests
java -cp ../bin szoftlab4.Main < teszteset.in
\end{lstlisting}
Ez azért lehet érdekes, mert ilyenkor látjuk a program kimenetét, nem csak a teszter mondja, hogy rendben lefutott a teszt.
A tesztesetek egy .in és egy .out fájlból állnak, a .in a bemenetet tartalmazza, a .out meg az elvárt kimenetet.

\section{Tesztek jegyzőkönyvei}

\subsection{attackdamagegem}

\tesztok{Tallér Bátor}{2014.04.22.}

\subsection{attackone}

\tesztok{Tallér Bátor}{2014.04.22.}

\tesztfail{Tallér Bátor}{2014.04.22.}{Nem támadta meg az ellenséget}{Nem rakja be a projectile-t a projectiles listába}{Hozzáadjuk a listához a létrejövő projectilet. }

\subsection{buildobstacle}

\tesztok{Tallér Bátor}{2014.04.22.}

\subsection{buildobstaclewrong}

\tesztok{Tallér Bátor}{2014.04.22.}

\subsection{buildtower}

\tesztok{Tallér Bátor}{2014.04.22.}

\subsection{buildtowerwrong}

\tesztok{Tallér Bátor}{2014.04.22.}

\subsection{critical}

\tesztok{Tallér Bátor}{2014.04.22.}
\subsection{elagazodasbalra}

\tesztok{Török Attila}{2014.04.22.}

\subsection{elagazodasjobbra}

\tesztok{Török Attila}{2014.04.22.}
\tesztfail{Török Attila}{2014.04.22.}{Az ellenség továbbra is véletlenszerűen haladt.}{Rossz map file.}{Map file átírása.}

\subsection{enemywin}

\tesztok{Török Attila}{2014.04.22.}

\subsection{fog}

\tesztok{Török Attila}{2014.04.22.}

\subsection{loadmap}

\tesztok{Török Attila}{2014.04.22.}

\subsection{noobstacle}

\tesztok{Török Attila}{2014.04.22.}

\subsection{obstacle}

\tesztok{Török Attila}{2014.04.22.}

\subsection{oneenemymove}

\tesztok{Török Attila}{2014.04.22.}

%\comment{Az alábbi táblázatot a megismételt (hibás) tesztek esetén kell kitölteni minden ismétléshez egyszer. Ha szükséges, akkor a valós kimenet is mellékelhető mint a teszt eredménye.}
\section{Változások - Tesztek}
\subsection{attack\_damage\_gem}
Létrehoz egy tornyot olyan távolságban az úttól, hogy ne legyen a hatósugarában, de egy piros varázskővel már elérje.
Majd rárak egy piros varázskövet a toronyra.\\
Bemenet:\\
loadMap test.map\\
loadMission attack.mission\\
buildTower 7 1\\
enchant red 7 1\\
listTowers\\
step 34\\
listEnemies\\
step 1\\
listEnemies\\
\\
exit\\ \\
Elvárt kimenet:\\
(7.0;1.0) red \\
1 75.0 (8.6;8.6) \\
1 45.0 (8.8;8.8) \\
\\
\subsection{attack\_one}
A teszt misszióban egy ellenség van. A teszt épít egy tornyot, majd megfelelő mennyiségű lépés után a torony elindít egy projectile-t az ellenség felé. \\
Bemenet:\\
loadMap test.map\\
loadMission attack.mission\\
buildTower 7 1\\
listTowers\\
step 30\\
listEnemies\\
listProjectiles\\
step 1\\
listEnemies\\
listProjectiles\\
step 1\\
listEnemies\\
listProjectiles\\
step 1\\
listEnemies\\
listProjectiles\\
step 1\\
listEnemies\\
listProjectiles\\
step 1\\
listEnemies\\
listProjectiles\\
\\
exit\\
\\
Elvárt kimenet:\\\\
(7.0;1.0) -\\
1 75.0 (7.5;7.5)\\
1 75.0 (7.8;7.8)\\
1 75.0 (8.0;8.0)\\
(7.0;1.0) 1 false\\
1 75.0 (8.3;8.3)\\
(7.6;4.3) 1 false\\
1 75.0 (8.6;8.6)\\
(8.3;7.5) 1 false\\
1 55.0 (8.8;8.8)\\

\subsection{buildobstacle}
Épít egy akadályt útra, majd kilistázza.\\
Bemenet:\\
loadMap test.map\\
loadMission test.mission\\
buildObstacle 5 5\\
listObstacles\\
exit\\
Elvárt kimenet:\\
(5.0;5.0) -
\subsection{buildobstacle\_wrong}
Azt teszteli, hogy akadályt nem lehet nem útra rakni.\\
Bemenet:\\
loadMap test.map\\
loadMission test.mission\\
buildObstacle 5 20\\
listObstacles\\
exit\\
Elvárt kimenet:\\
\\
\subsection{buildtower}
Torony építését teszteli.\\
Bemenet:\\
loadMap test.map\\
loadMission test.mission\\
buildTower 5 20\\
listTowers\\
exit\\
\\
Elvárt kimenet:\\
(5.0;20.0) -\\
\subsection{buildtower\_wrong}
Azt teszteli, hogy tornyot nem lehet útra rakni.\\
Bemenet:\\
loadMap test.map\\
loadMission test.mission\\
buildTower 5 5\\
listTowers\\
\\
exit\\
Elvárt kimenet:\\
\\
\subsection{critical}
Szétvágó lövedéket teszteli.\\
Bemenet:\\
loadMap test.map\\
loadMission attack.mission\\
setCritical 1\\
buildTower 7 1\\
listTowers\\
step 30\\
listEnemies\\
listProjectiles\\
step 1\\
listEnemies\\
listProjectiles\\
step 1\\
listEnemies\\
listProjectiles\\
step 1\\
listEnemies\\
listProjectiles\\
step 1\\
listEnemies\\
listProjectiles\\
step 1\\
listEnemies\\
listProjectiles\\
\\
exit\\
Elvárt kimenet:\\
(7.0;1.0) -\\
1 75.0 (7.5;7.5)\\
1 75.0 (7.8;7.8)\\
1 75.0 (8.0;8.0)\\
(7.0;1.0) 1 true\\
1 75.0 (8.3;8.3)\\
(7.6;4.3) 1 true\\
1 75.0 (8.6;8.6)\\
(8.3;7.5) 1 true\\
1 55.0 (8.8;8.8)\\
2 55.0 (8.8;8.8)\\
\subsection{elagazodas\_balra}
Elágazás tesztelése.
Elágazásnál, ahol két irányba mehet balra megy (igazából inkább lefelé).\\
Bemenet:\\
loadMap elagazodas.map\\
loadMission one\_enemy.mission\\
step 1\\
setWaypoint 0 3\\
step 70\\
listEnemies\\
exit\\
\\
Elvárt kimenet:\\
0 100.0 (11.1;18.5)\\
\subsection{elagazodas\_jobbra}
Elágazás tesztelése.
Elágazásnál, ahol két irányba mehet jobbra megy.\\
Bemenet:\\
loadMap elagazodas.map\\
loadMission one\_enemy.mission\\
step 1\\
setWaypoint 0 4\\
step 70\\
listEnemies\\
exit\\
\\
Elvárt kimenet:\\
0 100.0 (18.5;11.1)\\
\subsection{enemy\_win}
Az ellenség nyerését teszteli.\\
Bemenet:\\
loadMap test.map\\
loadMission test.mission\\
step 100\\
listEnemies\\
exit\\
\\
Elvárt kimenet:\\
Enemy winz\\
\subsection{fog}
Ködöt teszteli. A teszt program olyan tornyot épít, ami rálát egy útra, de a köd bekapcsolásával már nem. \\
Bemenet:\\
loadMap test.map\\
loadMission attack.mission\\
setFog 1\\
buildTower 7 1\\
listTowers\\
step 30\\
listEnemies\\
listProjectiles\\
step 1\\
listEnemies\\
listProjectiles\\
step 1\\
listEnemies\\
listProjectiles\\
step 1\\
listEnemies\\
listProjectiles\\
step 1\\
listEnemies\\
listProjectiles\\
step 1\\
listEnemies\\
listProjectiles\\
\\
exit\\
\\
Elvárt kimenet:\\
(7.0;1.0) -\\
1 75.0 (7.5;7.5)\\
1 75.0 (7.8;7.8)\\
1 75.0 (8.0;8.0)\\
1 75.0 (8.3;8.3)\\
1 75.0 (8.6;8.6)\\
1 75.0 (8.8;8.8)\\
\subsection{loadmap}
Pályabetöltés tesztelése.\\
Bemenet:\\
loadMap test.map\\
loadMission test.mission\\
exit\\
\\
Elvárt kimenet: \\
\\
\subsection{no\_obstacle}
obstacle teszesettel való összehasonlításra. A két teszt ugyanazt végzi, csak az obstacle-ben van egy akadály.\\
Bemenet:\\
loadMap test.map\\
loadMission attack.mission\\
listTowers\\
step 15\\
listEnemies\\
listProjectiles\\
step 5\\
listEnemies\\
listProjectiles\\
step 5\\
listEnemies\\
listProjectiles\\
step 5\\
listEnemies\\
listProjectiles\\
step 5\\
listEnemies\\
listProjectiles\\
step 5\\
listEnemies\\
listProjectiles\\
step 5\\
listEnemies\\
listProjectiles\\
step 5\\
listEnemies\\
listProjectiles\\
\\
exit\\
\\
Elvárt kimenet:\\
1 75.0 (3.6;3.6)\\
1 75.0 (4.9;4.9)\\
1 75.0 (6.2;6.2)\\
1 75.0 (7.5;7.5)\\
1 75.0 (8.8;8.8)\\
1 75.0 (10.1;10.1)\\
1 75.0 (11.4;11.4)\\
1 75.0 (12.7;12.7)\\
\subsection{obstacle}
Akadály tesztelése. A pályára épített akadály lassítja az ellenséget.\\
Bemenet:\\
loadMap test.map\\
loadMission attack.mission\\
buildObstacle 10 10\\
listTowers\\
step 15\\
listEnemies\\
listProjectiles\\
step 5\\
listEnemies\\
listProjectiles\\
step 5\\
listEnemies\\
listProjectiles\\
step 5\\
listEnemies\\
listProjectiles\\
step 5\\
listEnemies\\
listProjectiles\\
step 5\\
listEnemies\\
listProjectiles\\
step 5\\
listEnemies\\
listProjectiles\\
step 5\\
listEnemies\\
listProjectiles\\
\\
exit\\
\\
Elvárt kimenet:\\
1 75.0 (3.6;3.6)\\
1 75.0 (4.9;4.9)\\
1 75.0 (6.2;6.2)\\
1 75.0 (7.0;7.0)\\
1 75.0 (7.6;7.6)\\
1 75.0 (8.3;8.3)\\
1 75.0 (8.9;8.9)\\
1 75.0 (9.6;9.6)\\
\\
\subsection{one\_enemy\_move}
Egy ellenség mozgatásánal a tesztelése.\\
Bemenet:\\
loadMap test.map\\
loadMission one\_enemy.mission\\
step 31\\
listEnemies\\
exit\\
\\
Elvárt kimenet:\\
0 100.0 (7.1;7.1)\\

\subsection{elagazodas.map tartalma}
<map>\\
\phantom{pina}<name>Test\_Multi\_Waypoint</name>\\
\phantom{pina}<waypoint>\\
\phantom{pina}\phantom{pina}<id>1</id>\\
\phantom{pina}\phantom{pina}<coords>\\
\phantom{pina}\phantom{pina}\phantom{pina}<x>0</x>\\
\phantom{pina}\phantom{pina}\phantom{pina}<y>0</y>\\
\phantom{pina}\phantom{pina}</coords>\\
\phantom{pina}</waypoint>\\
\phantom{pina}<waypoint>\\
\phantom{pina}\phantom{pina}<id>2</id>\\
\phantom{pina}\phantom{pina}<coords>\\
\phantom{pina}\phantom{pina}\phantom{pina}<x>10</x>\\
\phantom{pina}\phantom{pina}\phantom{pina}<y>10</y>\\
\phantom{pina}\phantom{pina}</coords>\\
\phantom{pina}</waypoint>\\
\phantom{pina}<waypoint>\\
\phantom{pina}\phantom{pina}<id>3</id>\\
\phantom{pina}\phantom{pina}<coords>\\
\phantom{pina}\phantom{pina}\phantom{pina}<x>10</x>\\
\phantom{pina}\phantom{pina}\phantom{pina}<y>20</y>\\
\phantom{pina}\phantom{pina}</coords>\\
\phantom{pina}</waypoint>\\
\phantom{pina}<waypoint>\\
\phantom{pina}\phantom{pina}<id>4</id>\\
\phantom{pina}\phantom{pina}<coords>\\
\phantom{pina}\phantom{pina}\phantom{pina}<x>20</x>\\
\phantom{pina}\phantom{pina}\phantom{pina}<y>10</y>\\
\phantom{pina}\phantom{pina}</coords>\\
\phantom{pina}</waypoint>\\
\phantom{pina}<waypoint>\\
\phantom{pina}\phantom{pina}<id>5</id>\\
\phantom{pina}\phantom{pina}<coords>\\
\phantom{pina}\phantom{pina}\phantom{pina}<x>20</x>\\
\phantom{pina}\phantom{pina}\phantom{pina}<y>20</y>\\
\phantom{pina}\phantom{pina}</coords>\\
\phantom{pina}</waypoint>\\
\\
\\
\phantom{pina}<route>\\
\phantom{pina}\phantom{pina}<a>1</a>\\
\phantom{pina}\phantom{pina}<b>2</b>\\
\phantom{pina}\phantom{pina}<chance>1</chance>\\
\phantom{pina}</route>\\
\phantom{pina}\\
\phantom{pina}<route>\\
\phantom{pina}\phantom{pina}<a>2</a>\\
\phantom{pina}\phantom{pina}<b>3</b>\\
\phantom{pina}\phantom{pina}<chance>0.5</chance>\\
\phantom{pina}</route>\\
\phantom{pina}<route>\\
\phantom{pina}\phantom{pina}<a>2</a>\\
\phantom{pina}\phantom{pina}<b>4</b>\\
\phantom{pina}\phantom{pina}<chance>0.5</chance>\\
\phantom{pina}</route>\\
\phantom{pina}<route>\\
\phantom{pina}\phantom{pina}<a>3</a>\\
\phantom{pina}\phantom{pina}<b>5</b>\\
\phantom{pina}\phantom{pina}<chance>1</chance>\\
\phantom{pina}</route>\\
\phantom{pina}<route>\\
\phantom{pina}\phantom{pina}<a>4</a>\\
\phantom{pina}\phantom{pina}<b>5</b>\\
\phantom{pina}\phantom{pina}<chance>1</chance>\\
\phantom{pina}</route>\\
</map>\\

\subsection{test.map tartalma}
<map>\\
\phantom{pina}<name>Test</name>\\
\phantom{pina}<waypoint>\\
\phantom{pina}\phantom{pina}<id>1</id>\\
\phantom{pina}\phantom{pina}<coords>\\
\phantom{pina}\phantom{pina}\phantom{pina}<x>0</x>\\
\phantom{pina}\phantom{pina}\phantom{pina}<y>0</y>\\
\phantom{pina}\phantom{pina}</coords>\\
\phantom{pina}</waypoint>\\
\phantom{pina}<waypoint>\\
\phantom{pina}\phantom{pina}<id>2</id>\\
\phantom{pina}\phantom{pina}<coords>\\
\phantom{pina}\phantom{pina}\phantom{pina}<x>100</x>\\
\phantom{pina}\phantom{pina}\phantom{pina}<y>100</y>\\
\phantom{pina}\phantom{pina}</coords>\\
\phantom{pina}</waypoint>\\
\phantom{pina}<route>\\
\phantom{pina}\phantom{pina}<a>1</a>\\
\phantom{pina}\phantom{pina}<b>2</b>\\
\phantom{pina}\phantom{pina}<chance>1</chance>\\
\phantom{pina}</route>\\
</map>\\

\subsection{attack.mission tartalma}
<mission>\\
\phantom{pina}<name>Test</name>\\
\phantom{pina}<enemy>\\
\phantom{pina}\phantom{pina}<id>1</id>\\
\phantom{pina}\phantom{pina}<waypointID>1</waypointID>\\
\phantom{pina}\phantom{pina}<type>hobbit</type>\\
\phantom{pina}\phantom{pina}<time>0</time>\\
\phantom{pina}</enemy>\\
</mission>\\
\subsection{one\_enemy.mission tartalma}
<mission>\\
\phantom{pina}<name>Test</name>\\
\phantom{pina}<enemy>\\
\phantom{pina}\phantom{pina}<id>0</id>\\
\phantom{pina}\phantom{pina}<waypointID>1</waypointID>\\
\phantom{pina}\phantom{pina}<type>human</type>\\
\phantom{pina}\phantom{pina}<time>0</time>\\
\phantom{pina}</enemy>\\
</mission>\\
\subsection{test.mission tartalma}
<mission>\\
\phantom{pina}<name>Test</name>\\
\phantom{pina}<enemy>\\
\phantom{pina}\phantom{pina}<id>1</id>\\
\phantom{pina}\phantom{pina}<waypointID>1</waypointID>\\
\phantom{pina}\phantom{pina}<type>hobbit</type>\\
\phantom{pina}\phantom{pina}<time>0</time>\\
\phantom{pina}</enemy>\\
\phantom{pina}<enemy>\\
\phantom{pina}\phantom{pina}<id>2</id>\\
\phantom{pina}\phantom{pina}<waypointID>1</waypointID>\\
\phantom{pina}\phantom{pina}<type>hobbit</type>\\
\phantom{pina}\phantom{pina}<time>0</time>\\
\phantom{pina}</enemy>\\
\phantom{pina}<enemy>\\
\phantom{pina}\phantom{pina}<id>3</id>\\
\phantom{pina}\phantom{pina}<waypointID>1</waypointID>\\
\phantom{pina}\phantom{pina}<type>hobbit</type>\\
\phantom{pina}\phantom{pina}<time>0</time>\\
\phantom{pina}</enemy>\\
\phantom{pina}<enemy>\\
\phantom{pina}\phantom{pina}<id>4</id>\\
\phantom{pina}\phantom{pina}<waypointID>1</waypointID>\\
\phantom{pina}\phantom{pina}<type>hobbit</type>\\
\phantom{pina}\phantom{pina}<time>0</time>\\
\phantom{pina}</enemy>\\
\phantom{pina}<enemy>\\
\phantom{pina}\phantom{pina}<id>5</id>\\
\phantom{pina}\phantom{pina}<waypointID>1</waypointID>\\
\phantom{pina}\phantom{pina}<type>hobbit</type>\\
\phantom{pina}\phantom{pina}<time>0</time>\\
\phantom{pina}</enemy>\\
</mission>\\
\section{Értékelés}
%\comment{A projekt kezdete óta az értékelésig eltelt időben tagokra bontva, százalékban.}

\begin{ertekeles}
\tag{Nusser} % Tag neve
{25}        % Munka szazalekban
\tag{Szabó}
{27}
\tag{Tallér}
{29}
\tag{Török}
{19}
\end{ertekeles}

