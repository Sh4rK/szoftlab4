% Szglab4
% ===========================================================================
%
\chapter{Prototípus beadása}

\thispagestyle{fancy}

\section{Fordítási és futtatási útmutató}
% \comment{A feltöltött program fordításával és futtatásával kapcsolatos útmutatás. Ennek tartalmaznia kell leltárszerűen az egyes fájlok pontos nevét, méretét byte-ban, keletkezési idejét, valamint azt, hogy a fájlban mi került megvalósításra.}

\subsection{Fájllista}

\begin{fajllista}

\fajl
{Enemy.java}
{3 473 byte}
{2014.03.31~10:16~}
{Az ellenségeket megvalósító osztály.}

\fajl
{EnemyType.java}
{1 042 byte}
{2014.03.31~10:16~}
{Az ellenségek típusait leíró osztály.}

\fajl
{Fog.java}
{513 byte}
{2014.04.21~11:46~}
{A köd viselkedését meghatározó osztály.}

\fajl
{Game.java}
{11 727 byte}
{2014.03.31~10:16~}
{A játék mechanikáját tartalmazó osztály.}

\fajl
{Gem.java}
{357 byte}
{2014.03.31~10:16~}
{A varázskövek közös absztrakt ősosztálya.}

\fajl
{Map.java}
{3 628 byte}
{2014.03.31~10:16~}
{Egy térképet tároló osztály.}

\fajl
{Mission.java}
{2 390 byte}
{2014.03.31~10:16~}
{Egy küldetés menetét leíró osztály.}

\fajl
{Obstacle.java}
{2 116 byte}
{2014.03.31~10:16~}
{Egy akadályt megvalósító osztály.}

\fajl
{ObstacleGem.java}
{821 byte}
{2014.03.31~10:16~}
{Az akadályokra rakható varázskő osztálya.}

\fajl
{Pair.java}
{225 byte}
{2014.04.21~13:24~}
{Egy egyszerű generikus pár.}

\fajl
{Projectile.java}
{1 228 byte}
{2014.03.31~10:16~}
{A lövedékeket működtető osztály.}

\fajl
{Spawn.java}
{231 byte}
{2014.04.22~6:59~}
{Egy ellenség megjelenését tároló osztály.}

\fajl
{SplitterProjectile.java}
{760 byte}
{2014.04.21~15:42~}
{Speciális, az ellenséget kettévágó Projectile.}

\fajl
{Tower.java}
{3 696 byte}
{2014.03.31~10:16~}
{Egy tornyot leíró osztály.}

\fajl
{TowerGem.java}
{928 byte}
{2014.03.31~10:16~}
{A tornyokat erősítő varázsköveket tárolja.}

\fajl
{Vector.java}
{1 356 byte}
{2014.03.31~10:16~}
{Egy két dimenziós valós vektor.}

\fajl
{Waypoint.java}
{2 319 byte}
{2014.03.31~10:16~}
{Az ellenségek útvonalainak egy csomópontja.}


\fajl
{Main.java}
{4 548 byte}
{2014.04.22~11:09~}
{A teszter segédprogram főosztálya.}


\end{fajllista}

\subsection{Fordítás}
%\comment{A fenti listában szereplő forrásfájlokból milyen műveletekkel lehet a bináris, futtatható kódot előállítani. Az előállításhoz csak a 2. Követelmények c. dokumentumban leírt környezetet szabad előírni.}

\lstset{escapeinside=`', xleftmargin=10pt, frame=single, basicstyle=\ttfamily\footnotesize, language=sh}
\begin{lstlisting}
javac -encoding utf8 -d bin src/szoftlab4/*.java src/Tester/*.java
\end{lstlisting}

\subsection{Futtatás}
%\comment{A futtatható kód elindításával kapcsolatos teendők leírása. Az indításhoz csak a 2. Követelmények c. dokumentumban leírt környezetet szabad előírni.}

\lstset{escapeinside=`', xleftmargin=10pt, frame=single, basicstyle=\ttfamily\footnotesize, language=sh}
Teszter indítása
\begin{lstlisting}
cd tests
java -cp ../bin Tester.Main
\end{lstlisting}

Program indítása, ekkor a standard bementetről lehet parancsokat adni a programnak
\begin{lstlisting}
cd bin
java szoftlab4.Main
\end{lstlisting}

Ilyenkor lehetséges előre elkészített teszteseteket is lefuttatni, pl így:
\begin{lstlisting}
cd tests
java -cp ../bin szoftlab4.Main < teszteset.in
\end{lstlisting}
Ez azért lehet érdekes, mert ilyenkor látjuk a program kimenetét, nem csak a teszter mondja, hogy rendben lefutott a teszt.
A tesztesetek egy .in és egy .out fájlból állnak, a .in a bemenetet tartalmazza, a .out meg az elvárt kimenetet.

\section{Tesztek jegyzőkönyvei}

\subsection{attackdamagegem}

\tesztok{Tallér Bátor}{2014.04.22.}

\subsection{attackone}

\tesztok{Tallér Bátor}{2014.04.22.}

\tesztfail{Tallér Bátor}{2014.04.22.}{Nem támadta meg az ellenséget}{Nem rakja be a projectile-t a projectiles listába}{Hozzáadjuk a listához a létrejövő projectilet. }

\subsection{buildobstacle}

\tesztok{Tallér Bátor}{2014.04.22.}

\subsection{buildobstaclewrong}

\tesztok{Tallér Bátor}{2014.04.22.}

\subsection{buildtower}

\tesztok{Tallér Bátor}{2014.04.22.}

\subsection{buildtowerwrong}

\tesztok{Tallér Bátor}{2014.04.22.}

\subsection{critical}

\tesztok{Tallér Bátor}{2014.04.22.}
\subsection{elagazodasbalra}

\tesztok{Török Attila}{2014.04.22.}

\subsection{elagazodasjobbra}

\tesztok{Török Attila}{2014.04.22.}
\tesztfail{Török Attila}{2014.04.22.}{Az ellenség továbbra is véletlenszerűen haladt.}{Rossz map file.}{Map file átírása.}

\subsection{enemywin}

\tesztok{Török Attila}{2014.04.22.}

\subsection{fog}

\tesztok{Török Attila}{2014.04.22.}

\subsection{loadmap}

\tesztok{Török Attila}{2014.04.22.}

\subsection{noobstacle}

\tesztok{Török Attila}{2014.04.22.}

\subsection{obstacle}

\tesztok{Török Attila}{2014.04.22.}

\subsection{oneenemymove}

\tesztok{Török Attila}{2014.04.22.}

%\comment{Az alábbi táblázatot a megismételt (hibás) tesztek esetén kell kitölteni minden ismétléshez egyszer. Ha szükséges, akkor a valós kimenet is mellékelhető mint a teszt eredménye.}
\section{Értékelés}
%\comment{A projekt kezdete óta az értékelésig eltelt időben tagokra bontva, százalékban.}

\begin{ertekeles}
\tag{Nusser} % Tag neve
{25}        % Munka szazalekban
\tag{Szabó}
{27}
\tag{Tallér}
{29}
\tag{Török}
{19}
\end{ertekeles}

